\chapter*{Úvod}
\addcontentsline{toc}{chapter}{Úvod}

Scientometria (ang. {\em scientometrics}) je vedný obor, ktorý sa zaoberá
hodnotením vedy, t.j. vedeckých publikácii, vedeckých pracovníkov a vedeckého
pokroku použitím matematických, štatistických metód.  Súbor týchto metód sa
nazýva {\em scientometrika}.  Hlavným aspektom, podľa ktorého sa hodnotia
vedecké práce sú citácie, t.j. referencie na iné publikácie, ktoré autor použil,
alebo chce na ne upozorniť.  Všeobecne je brané, že publikácia, ktorú cituje
viacej iných vedeckých článkov má väčší impakt (dopad).  To znamená, že práca je
populárna, používaná, pretože je kvalitná a prínosná pre vedecký pokrok.
%poznamka

Ďalším spôsobom hodnotenia vedy je použitie ekonomických aspektov.  Počet a
hodnota grantov, ktoré daný pracovník, či inštitúcia dosiahli, alebo hodnota
praktického uplatnenia konkrétnych poznatkov.  Bohužiaľ tento spôsob hodnotenia
nemôže byť všeobecný a ani spravodlivý, pretože cieľom vedy niekedy nemusí
primárne vytvoriť prospešný a ekonomicky výhodný produkt, ale posunúť ľudské
poznanie.  Väčšina výskumu patrí do tzv.  základného výskumu, v ktorom sa
bezprostredne neočakáva možnosť aplikácie nadobudnutých poznatkov do
praxe. Mnohé z nich nie sú doteraz aplikovateľné a niektoré našli praktické
uplatnenie až po uplynutí niekoľko storočí (napr. matematické modely umelej
inteligencie).  V neposlednom rade predmetom výskumu je overenie hypotézy.
Vedec by mal očakávať, že výsledok výskumu bude vyvrátenie hypotézy a
automaticky neprinesie ekonomický úžitok, ale iba pokrok.  Každý vedec musí mať
na pamäti, že aj negatívny výsledok je výsledok.

Cieľom tejto práce je scientometrické hodnotenie publikačnej činnosti
zamestnancov Fakulty prírodných vied Univerzity sv. Cyrila a Metoda v Trnave.
Hodnotenie je vykonané kvantitatívne, počtom publikovaných prác a tiež
kvalitatívne pomocou tzv. citačných indexov vypočítaných programom {\em Publish
  or Perish}.  Vstupné dáta do hodnotenia boli získané z najväčších a
najvýznamnejších citačných databáz {\em Elsevier Scopus} a {\em Thomson Reuters
  Web of Science}.

%%% Local Variables:
%%% TeX-master: "diplomovka"
%%% End:
