\chapter*{Príloha}
\addcontentsline{toc}{chapter}{Príloha}

\section*{Popis našich programov na scientometrickú analýzu}


\subsection*{Program \emph{scientometry-data-proc}}
%\section*{Konfiguračný súbor programu \emph{scientometry-data-proc}}

Program \emph{scientometry-data-proc} sme vyvinuli ako univerzálnu platformu na
spracovanie scientometrických dát.  Samotný program je napísaný v programovacom
jazyku \emph{Python} verzie 2.7 a pre pre svoju činnosť potrebuje mať nadefinované
parametre v textovom konfiguračnom súbore \verb|config.yaml| v značkovacom
jazyku YAML\footnote{\url{http://yaml.org/}}. Pre účely tejto práce sme použili
nasledujúci konfiguračný súbor:

\begin{source}
\begin{verbatim}
defaults:
  output-dir:

all-publications-data:
  class: PublicationsData
  source:
    Scopus: all-scopus-{date}.csv
    WoS: all-wos-{date}.csv
  years: 2000-2016
  select: [ Year, Scopus, WoS ]

all-citations-data:
  class: CitationsData
  source:
    Scopus: all-scopus-{date}.csv
    WoS: all-wos-{date}.csv
  years: 2000-2016
  select: [ Year, Scopus, WoS ]

journals-data:
  class: JournalsData
  source: all-merged-{date}.csv
  journal-catalog: journal-catalog-{date}.csv
  select: [ Source title, Papers, ISI JIF, SJR, Scimago h-index, CiteScore, SNIP, Notes ]

results-data:
  class: ResultsData
  source: results-all-{date}.csv
  groups: [ KB, KBt, KCh, KER, KBf, KAIM ]
  select: [ Group, Scopus, WoS, GS ]
\end{verbatim}
\end{source}

Jednotlivé sekcie súboru definujú nezávislé časti spracovania dát a ich
parametre. V sekcii \verb|defaults| sú definované implicitné parametre platné
pre všetky sekcie. Samotný proces spracovania dát definuje trieda
(\verb|class|). Zvyšok časti sekcie len definuje parametre pre vybranú triedu.

Program postupne vykoná nasledujúce operácie:

\begin{enumerate}
\item Načíta vstupné súbory obsahujúce dáta z citačných registrov Scopus a
  WoS. Spočíta v počet článkov, ktoré boli publikované v danom roku. Následne
  vygeneruje súbor \verb|all-publications-data.csv|, ktorý bude obsahovať
  tabuľku so stĺpcami definovanými parametrom \verb|select| a riadkami
  definovanými parametrom \verb|years|.
\item Načíta vstupné súbory obsahujúce dáta z citačných registrov Scopus a
  WoS. Spočíta v počet citácii na publikácie, ktoré boli publikované v rovnakom
  roku. Následne vygeneruje súbor \verb|all-citations-data.csv|, ktorý bude
  obsahovať tabuľku so stĺpcami definovanými parametrom \verb|select| a riadkami
  definovanými parametrom \verb|years|.
\item Načíta vstupný súbor obsahujúci ručne zlúčené dáta zo všetkých citačných
  registrov a podľa ISSN spočíta počet článkov publikovaných v jednotlivých
  časopisoch. Tieto hodnoty spojí s ručne zostaveným katalógom všetkých
  dostupných časopisov a vytvorí jednu tabuľku, ktorú uloží do súboru
  \verb|journals-data.csv|.
\item Načíta upravenú verziu výstupého súboru programu \emph{Publish or Perish}.
  Dáta zo súboru rozdelí do menších tabuliek\,--\,pre každý citačný indikátor
  jednu. Všetky výstupné tabuľky budú obsahovať stĺpce definované parametrom
  \verb|select| a riadky definované parametrom \verb|groups|. Budú uložené do
  súborov vo formáte CSV s prefixom \verb|results-data.csv|.
\end{enumerate}


\subsection*{Program \emph{scientometry-plot-gen}}

Program \emph{scientometry-plot-gen} slúži na automatizáciu vytvárania
stĺpcových diagramov z dátových súborov vo formáte CSV. Program je napísaný v
jazyku \emph{Python} verzie 2.7, na kreslenie grafov využíva knižnicu
\emph{matplotlib} a vytvára súbory vo formáte \texttt{.png}.

Cieľom bolo spracovanie dát čo najviac automatizovať, preto je program
\emph{scientometry-plot-gen} navrhnutý tak, aby mohol spracovávať výstupné
súbory programu \emph{scientometry-data-proc} bez nutnosti ručných zásahov.  I
napriek našej snahe o multiplatformovosť, program v súčasnosti funguje iba v
operačných systémoch typu UNIX (GNU/Linux, MacOS, BSD). Tento nedostatok nám
našťastie nijak nebránil vo využití programu pre účely tejto práce.

Metadáta a parametre každého grafu je potrebné špecifikovať pomocou textového
súboru \verb|plot-metadata.yaml|, ktorý využíva syntax značkovacieho jazyka
YAML. V tomto súbore je možné definovať metadáta pre ľubovoľný počet
grafov, čo umožňuje rýchle generovanie celých sád grafov.

Na základe nasledujúceho konfiguračného súboru vygeneroval
program \emph{scientometry-plot-gen} v jednej dávke všetky grafy, ktoré sú
súčasťou tejto práce.

\begin{source}
\begin{verbatim}


defaults:
  format: png
  resolution: 150 # [dpi]
  figsize:
    - 25.4 # width [cm]
    - 14.1 # height [cm]
  suptitle_fontsize: 14
  title_fontsize: 12
  ticklabel_fontsize: 11
  axislabel_fontsize: 11
  legend_fontsize: 11
  legend_loc: best
  suptitle: ''
  xlabel: Katedra
  title_y: 1.004
  barwidth: 0.2
  barcolors:
    - blue
    - red
    - green
  legend:
    - Scopus
    - WoS
    - GS
  ymax: best

all-publications-data:
  title: Vývoj publikačnej činnosti FPV v období 2000-2016
  xlabel: Rok publikovania
  ylabel: Počet publikácií
  barwidth: 0.3
  legend_loc: upper left
  legend:
    - Scopus
    - WoS

all-citations-data:
  title: Citovanosť publikácii FPV v období 2000-2016
  xlabel: Rok publikovania
  ylabel: Počet citácií
  barwidth: 0.3
  ymax: 350
  legend:
    - Scopus
    - WoS

results-data-papers:
  title:  Celkový počet publikácií jednotlivých katedier FPV
  ylabel: Počet publikácií

results-data-citations:
  title:  Celkový počet citácií jednotlivých katedier FPV
  ylabel: Počet citácií

results-data-papers_author:
  title: Medián počtu publikácií na autora pre jednotlivé katedry FPV
  ylabel: Medián počtu publikácií na autora

results-data-cites_paper:
  title: Medián citácií na publikáciu pre jednotlivé katedry FPV
  ylabel: Medián citácií na publikáciu

results-data-e_index:
  title: Medián $e$-indexu pre jednotlivé katedry FPV
  ylabel: Medián $e$-indexu
  ymax: 14

results-data-h_index:
  title: Medián $h$-indexu pre jednotlivé katedry FPV
  ylabel: Medián $h$-indexu
  ymax: 12

results-data-g_index:
  title: Medián $g$-indexu pre jednotlivé katediry FPV
  ylabel: Medián $g$-indexu
  ymax: 20

results-data-hi_index:
  title: Medián $h_{\mathrm{I}}$-indexu pre jednotlivé katedry FPV
  ylabel: Medián $h_{\mathrm{I}}$-indexu

results-data-hi_norm:
  title: Medián $h_{\mathrm{I, norm}}$ pre jednotlivé katedry FPV
  ylabel: Medián $h_{\mathrm{I, norm}}$

results-data-awcr:
  title: Medián AWCR pre jednotlivé katedry FPV
  ylabel: Medián AWCR
  ymax: 100

results-data-aw_index:
  title: Medián $\mathit{AW}$-indexu pre jednotlivé katedry FPV
  ylabel: Medián $\mathit{AW}$-indexu
  ymax: 10

results-data-hc_index:
  title: 'Medián $h^{\mathrm{c}}\!$-indexu pre jednotlivé katedry FPV'
  ylabel: 'Medián $h^{\mathrm{c}}\!$-indexu'
  ymax: 10

results-data-hm_index:
  title: Medián $h_{\mathrm{m}}$-indexu pre jednotlivé katedry FPV
  ylabel: Medián $h_{\mathrm{m}}$-indexu
  ymax: 5
\end{verbatim}
\end{source}

%%% Local Variables:
%%% TeX-master: "diplomovka"
%%% End:
