\chapter*{Úvod}
\addcontentsline{toc}{chapter}{Úvod}

Scientometria (ang. \emph{scientometrics}) je vedný obor, ktorý sa zaoberá
hodnotením vedy, t.j. vedeckých publikácii, vedeckých pracovníkov a vedeckého
pokroku použitím matematických, štatistických metód.  Súbor týchto metód sa
nazýva \emph{scientometrika}.  Hlavným aspektom, podľa ktorého sa hodnotia
vedecké práce sú citácie, t.j. referencie na iné publikácie, ktoré autor
použil, alebo chce na ne upozorniť.  Všeobecne sa berie, že publikácia, ktorú
cituje viacej iných vedeckých článkov má väčší impakt (dopad).  To znamená, že
práca je populárna, používaná, pretože je kvalitná a prínosná pre vedecký
pokrok.

Ďalším spôsobom hodnotenia vedy je použitie ekonomických aspektov.  Počet a
hodnota grantov, ktoré daný pracovník, či inštitúcia dosiahli, alebo hodnota
praktického uplatnenia konkrétnych poznatkov.  Bohužiaľ tento spôsob hodnotenia
nemôže byť všeobecný a ani spravodlivý, pretože cieľom vedy niekedy nemusí
primárne vytvoriť prospešný a ekonomicky výhodný produkt, ale posunúť ľudské
poznanie.  Väčšina výskumu patrí do tzv.  základného výskumu, v~ktorom sa
bezprostredne neočakáva možnosť aplikácie nadobudnutých poznatkov do praxe.
Mnohé z~nich nie sú doteraz aplikovateľné a niektoré našli praktické uplatnenie
až po uplynutí niekoľko storočí (napr. matematické modely umelej inteligencie).
V~neposlednom rade predmetom výskumu je overenie hypotézy.  Vedec by mal
očakávať, že výsledok výskumu bude vyvrátenie hypotézy a automaticky neprinesie
ekonomický úžitok, ale iba pokrok.

Cieľom tejto práce je scientometrické hodnotenie publikačnej činnosti Fakulty
prírodných vied Univerzity sv. Cyrila a Metoda v~Trnave.  Hodnotenie je
vykonané kvantitatívne, počtom publikovaných prác a tiež kvalitatívne pomocou
tzv. citačných indikátorom vypočítaných programom \emph{Publish or Perish}.
Vstupné dáta do hodnotenia boli získané z~najväčších a najvýznamnejších
citačných databáz \emph{Elsevier Scopus}, \emph{Thomson Reuters Web of Science}
a \emph{Google Scholar}

Kapitola 1 obsahuje stručné vysvetlenie najdôležitejších súčastí oboru
 scientometria. V~rámci tejto časti vysvetľujeme základné pojmi v~obore.
Následne niečo povieme o~najvýznamnejších citačných databázach a v~poslednej
časti tejto kapitoly sa venujem citačným indikátorom, ktoré sme použili
v~tejto práci.

V~prvej časti kapitoly 2 sa sústreďujeme  proces získavania bibliografických
dát na scientometrickú analýzu z~citačných databáz \emph{Scopus}, \emph{Web of
Science} a \emph{Google Scholar}.  Potom opisujeme spracovanie dát pomocou
scientometrického programu \emph{Publish or Perish} a vlastných skriptov.

Kapitola 3 definuje hlavný cieľ a čiastkové ciele, ktoré sú potrebné dosiahnuť
pre dokončenie práce.

V~kapitole 4 sú diskutované a zhodnocované  výsledky za pomoci prehľadných
tabuliek a grafov.

%%% Local Variables:
%%% TeX-master: "diplomovka"
%%% ispell-local-dictionary: "slovak"
%%% End:
