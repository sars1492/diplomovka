\chapter*{Úvod}
\addcontentsline{toc}{chapter}{Úvod}

Scientometria (ang.~\emph{scientometrics}) je vedný obor, ktorý sa zaoberá
hodnotením vedy, t.\,j.~vedeckých publikácii, vedeckých pracovníkov a~vedeckého
pokroku použitím matematických, štatistických metód.  Súbor týchto metód sa
nazýva \emph{scientometrika}.  Hlavným aspektom, podľa ktorého sa hodnotia
vedecké práce sú citácie, t.\,j.~referencie na iné publikácie, ktoré autor
použil, alebo chce na ne upozorniť.  Všeobecne sa berie, že publikácia, ktorú
cituje viacej iných vedeckých článkov má väčší impakt (dopad).  To znamená, že
práca je populárna, používaná, pretože je kvalitná a~prínosná pre vedecký
pokrok.

Ďalším spôsobom hodnotenia vedy je použitie ekonomických aspektov.  Počet
a~hodnota grantov, ktoré daný pracovník, či inštitúcia dosiahli, alebo hodnota
praktického uplatnenia konkrétnych poznatkov.  Bohužiaľ tento spôsob hodnotenia
nemôže byť všeobecný a~ani spravodlivý, pretože cieľom vedy niekedy nemusí
primárne vytvoriť prospešný a~ekonomicky výhodný produkt, ale posunúť ľudské
poznanie.  Väčšina výskumu patrí do tzv.~základného výskumu, v~ktorom sa
bezprostredne neočakáva možnosť aplikácie nadobudnutých poznatkov v praxi.
Mnohé z~nich nie sú doteraz aplikovateľné a~niektoré našli praktické uplatnenie
až po uplynutí niekoľkých storočí (napr.~matematické modely umelej
inteligencie).  V~neposlednom rade predmetom výskumu je overenie hypotézy.
Vedec by mal očakávať, že výsledok výskumu bude vyvrátenie hypotézy
a~automaticky neprinesie ekonomický úžitok, ale iba pokrok.

Cieľom tejto práce je scientometrické hodnotenie publikačnej činnosti Fakulty
prírodných vied Univerzity sv.~Cyrila a~Metoda v~Trnave.  Hodnotenie je
vykonané kvantitatívne, počtom publikovaných prác a~tiež kvalitatívne pomocou
tzv.~citačných indikátorov vypočítaných programom \emph{Publish or Perish}.
Vstupné dáta do hodnotenia boli získané z~najväčších a~najvýznamnejších
citačných databáz \emph{Elsevier Scopus}, \emph{Thomson Reuters Web of Science}
a \emph{Google Scholar}.

Kapitola~\ref{chap:review} obsahuje stručné zhrnutie najdôležitejších súčastí
modernej scientometrie.  V~úvode vysvetlíme základné pojmy a~poskytneme prehľad
najdôležitejších citačných databáz.  Vo zvyšnej časti sa zameriame na citačné
indikátory, ktoré sme použili v~tejto práci.

Kapitola~\ref{chap:objectives} definuje hlavný cieľ a~čiastkové ciele, ktoré sú
potrebné dosiahnuť pre dokončenie tejto práce.

V~prvej časti Kapitoly~\ref{chap:methods} sa sústreďujeme na proces získavania
bibliografických dát na scientometrickú analýzu z~citačných databáz
\emph{Scopus}, \emph{Web of Science} a~\emph{Google Scholar}.  V~záverečnej
časti popisujeme spracovanie dát pomocou scientometrického programu
\emph{Publish or Perish} a~vlastných programov.

V~Kapitole~\ref{chap:results} diskutujeme výsledky a~zhodnocujeme ich za pomoci
prehľadných tabuliek a~grafov.

%%% Local Variables:
%%% TeX-master: "diplomovka"
%%% ispell-local-dictionary: "slovak"
%%% End:
