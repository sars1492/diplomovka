\chapter*{Záver}
\addcontentsline{toc}{chapter}{Záver}

Scientometria je hodnotenie vedy, vedeckých publikácii, vedeckých pracovníkov
a~vedeckého pokroku použitím matematických, štatistických metód.  Hlavným zdrojom
na scientometrickú analýzu sú tzv.~bibliografické záznamy, ktoré obsahujú
informácie o~danej publikácii (napr.~autor, názov, časopis) a~počet citácií na
tento dokument (prípadne odkazy na citujúce publikácie).  Tieto záznamy sú
uložené v~databázach citačných registrov.  Z~nich najznámejšie sú \emph{Scopus}
(vydavateľstvo \emph{Elsevier}), \emph{Web of Science} (mediálny gigant
\emph{Thomson Reuters}) a~\emph{Google Scholar}.

Cieľom tejto práce bolo zhodnotiť Fakultu prírodných vied sv.~Cyrila a~Metoda
v~Trnave (FPV UCM v~Trnave) s~použitím rôznych scientometrických metód na
bibliografické dáta.  Tieto dáta boli získané z~citačných registrov
\emph{Scopus}, \emph{Web of Science} a~\emph{Google Scholar}.  Najväčšiu časť
dát tvoria všetky dostupné bibliografické záznamy vedcov zamestnaných na Fakulte
prírodných vied.  Ostatné dáta boli získané z~databáz \emph{Scopus} a~\emph{Web
  of Science} podľa príslušnosti autorov (ang.~\emph{affiliation}) k~fakulte.
Z~týchto bibliografických dát sme zhodnotili vývoj publikačnej činnosti
a~citovanosti v~čase, a~zhotovili tabuľku početnosti týchto dát v~odborných
časopisoch.  Dáta jednotlivých pracovníkov sme scientometricky zanalyzovali,
rozčlenili do jednotlivých katedier a~urobili celkovú scientometrickú analýzu
jednotlivých katedier.

Z~technických dôvodov sme do analýz nezahrnuli odstránenie autocitácií.  Táto
práca tiež nezahŕňa výpočet spoluautorských sietí, pretože by tak presiahla
predpísaný rozsah.  I~napriek tomu môžeme konštatovať, že sa nám v~rámci
možností podarilo splniť stanovené ciele.  Táto práca tak predstavuje prvú
komplexnejšiu scientometrickú analýzu Fakultu prírodných vied sv.~Cyrila
a~Metoda v~Trnave a~môže sa stať užitočným podkladom pre ďalšie práce tohto typu.

%%% Local Variables:
%%% TeX-master: "diplomovka"
%%% ispell-local-dictionary: "slovak"
%%% End:
