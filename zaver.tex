\chapter*{Záver}
\addcontentsline{toc}{chapter}{Záver}

Scientometria je hodnotenie vedy, vedeckých publikácii, vedeckých pracovníkov a
vedeckého pokroku použitím matematických, štatistických metód.  Hlavným
aspektom, podľa ktorého sa hodnotia vedecké práce sú citácie, t.j. referencie na
iné publikácie, ktoré autor použil, alebo chce na ne upozorniť.  Všeobecne je
brané, že publikácia, ktorú cituje viacej iných vedeckých článkov má väčší
impakt.  To znamená, že práca je populárna, pretože je kvalitná a prínosná pre
vedecký pokrok.

Ďalším spôsobom hodnotením vedy je použitie ekonomických aspektov.  Počet a
hodnota grantov, ktoré daný pracovník, či inštitúcia dosiahla, alebo hodnota
praktického uplatnenia konkrétnych poznatkov.  Bohužiaľ tento spôsob hodnotenia
nemôže byť všeobecný a ani spravodlivý.  Pretože cieľom vedy nie je vytvoriť
prospešný a ekonomicky výhodný produkt, ale posunúť ľudské poznanie.  Väčšina
výskumu patrí do tzv. základného výskumu, u~ktorého sa neočakáva možnosť
aplikácie nadobudnutých poznatkov do praxe.  Mnohé z~nich nie sú doteraz
aplikovateľné a niektoré našli praktické uplatnenie až po uplynutí niekoľko
storočí (napr. matematické modely umelej inteligencie).  V~neposlednom rade
predmetom výskumu je overenie hypotézy.  Vedec by mal očakávať, že výsledok
výskumu bude vyvrátenie hypotézy a automaticky neprinesie ekonomický úžitok, ale
iba pokrok.  Každý vedec musí mať na pamäti, že aj negatívny výsledok je
výsledok.

%%% Local Variables:
%%% TeX-master: "diplomovka"
%%% End: