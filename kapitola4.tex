\chapter{Výsledky a diskusia}

\section{Vývoj publikačnej činnosti a citovanosti článkov FPV}

Obrázok~\ref{fig:all.publications.plot} vyjadruje grafické znázornenie množstva
publikovaných článkov všetkými pracovníkmi Fakulty prírodných vied
v~jednotlivých rokoch z~citačných registrov Scopus a WoS (odlíšené farbami stĺpcov).
Je prirodzené, že pracovníci slovenskej univerzity viac článkov publikujú
v~európskych časopisoch, pre ktoré má Scopus väčšie pokrytie než WoS.  Čo môžeme
vidieť ako rozdiel výšky stĺpcov v~období 2004--2010.  Publikačný skok WoS od roku
2012 je spôsobený obsiahnutím šesťnástich kapitol z~knihy \emph{Handbook of
  Magentochemical Formulae} od doc.\,J.\,Boču z~Katedry chémie, ktoré nie sú
citované (knihy sú o~mnoho menej citované ako vedecké články).  Zvýšenie počtu
publikácií vo WoS v~roku 2015 je dôsledkom tlaku na získanie akreditácie.  Pre
akreditačnú komisiu komisiu sú najdôležitejšie časopisy indexované vo WoS.

\begin{figure}
  \centering
  \includegraphics[width=\textwidth]{plot-all-publications-data.png}
  \caption{Vývoj publikačnej činnosti pracovníkov FPV v~období 2000--2016.}
  \label{fig:all.publications.plot}
\end{figure}

Graf na Obrázku~\ref{fig:all.citations.plot} zobrazuje celkový počet citácií
článkov publikovaných v~danom roku.  Rozdiel v~citáciách medzi dátami z~Scopusu
a WoS sú viditeľné hlavne už spomínanom období 2004--2010.  Extrémny rozdiel
citácií na publikácie z~roku 2005 je spôsobený, že v~citačnom registri Scopus je
obsiahnutých 16 článkov z~roku 2005, ktorých 15 sú dobre citované.  WoS obsahuje
iba dva články z~tohto roku: ten necitovaný zo Scopusu a publikáciu, ktorá je
obsiahnutá iba v~databáze z~WoS, ale tiež bez citácií. V~roku 2015 vidíme
dvojnásobný nárast počtu citácií vo WoS oproti Scopusu.  Toto je spôsobené nárastom 
publikovania do časopisov, ktoré sú indexované vo WoS ako následok na tlak na
získanie akreditácie.

\begin{figure}
  \centering
  \includegraphics[width=\textwidth]{plot-all-citations-data.png}
  \caption{Citovanosť článkov všetkých pracovníkov FPV za obdobie
    2000--2016.}
  \label{fig:all.citations.plot}
\end{figure}


\section{Scientometrické hodnotenie katedier FPV}

Hlavnou časťou tejto práce bola scientometrická analýza jednotlivých katedier
podľa \citet{Kazakis2014a} a \citet{Kazakis2014b,Kazakis2015}. Metodika zahrňuje
výpočet citačných indikátorov zvlášť pre každého vedeckého pracovníka
z~danej inštitúcie. Indikátorom pre katedru predstavuje priemer a medián
hodnôt citačných indikátorov zamestnancov katedry.
Distribúcia indikátorov skupiny pracovníkov nie je normálna (tzv. gausovská),
a~ani sa neblíži k~normálnej. Z~toho dôvodu si myslíme, že na túto analýzuje je
vhodnejšie použiť medián než aritmetický priemer.

V~Tabuľkách  \ref{tab:1-staff.results}, \ref{tab:2-staff.results},
\ref{tab:3-staff.results}, \ref{tab:4-staff.results}, \ref{tab:5-staff.results}
a \ref{tab:6-staff.results} sú uvedené výsledné hodnoty indikátorov každú
katedru FPV.  Namiesto mien katedier sme použili oficiálne skratky (viď Tabuľka
\ref{tab:department.review}).  V~Tabuľke  \ref{tab:1-staff.results} stĺpec $n$
zodpovedá počtu vedeckých zamestancov, ktorí boli zahrnutí do~dátového súboru.
V~stĺpcoch $P$ a $c$ uvádzame sumu všetkých článkov (Obrázok
\ref{fig:publications.plot}) a citácií (Obrázok \ref{fig:citations.plot})
v~dátom súbore.  Tieto informácie uvádzame pre ilustráciu, pretože Veľkosti
dátových súborov, z~ktorých dane indikátory boli počítané, určujú presnosť
štatistiky. Všetky grafy neobsahujú výsledné hodnoty Katedry odbornej
jazykovej prípravy, pretože boli príliš nízke, z~dôvodu nedostatku dát.

% Katedra oficiálné skratky katedier\,--\,KB: Katedra biológie; KBt: Katedra biotechnológií; KCh: Katedra chémie; KER: Katedra ekochémie a rádiobiológie; KBf: Katedra biofyziky; KAIM: Katedra aplikovanej informatiky a matematiky; Katedra odbornej jazykovej prípravy. \\
%  počet vedeckých pracovníkov danej katedry }

\begin{table}
\centering\small
  \caption[Hodnotenie FPV\,--\,počet publikácií na autora]{Scientometrické hodnotenie katedier FPV UCM v~Trnave\,--\,počet publikácií na autora.}
  \label{tab:1-staff.results}
\begin{tabularx}{\textwidth}{lXc@{\hspace{2.5em}}c@{\hspace{2.5em}}c@{\hspace{3.5em}}cccc}
  \toprule\noalign{\vspace{.3ex}}
         &        &   &     &      & \multicolumn{4}{c}{Počet publikácií na autora}  \\
 Katedra & Cit. register  & $n$   & $P$     &  $c$     & $\bar{x}$      & $\sigma$  & $\tilde{x}$   & MAD  \\[0.3ex]
\midrule\noalign{\vspace{.5ex}}
 KB   & Scopus & 14 & 380  & 5234  & 9,09          & 13,20 & 3,42  & 3,02 \\
      & WoS    & 13 & 392  & 4982  & 10,66         & 14,44 & 7,20  & 5,18 \\
      & GS     & 14 & 594  & 7289  & 16,88         & 20,77 & 8,92  & 7,42 \\[3ex]
 KBt  & Scopus & 11 & 356  & 11492 & 8,34          & 12,16 & 4,44  & 2,49 \\
      & WoS    & 11 & 351  & 3707  & 7,83          & 10,48 & 5,20  & 3,49 \\
      & GS     & 11 & 828  & 14901 & 23,15         & 24,90 & 12,10 & 5,60 \\[3ex]
 KCh  & Scopus & 18 & 1106 & 19248 & 15,42         & 21,10 & 5,34  & 4,32 \\
      & WoS    & 18 & 1120 & 19223 & 16,14         & 21,85 & 5,76  & 4,54 \\
      & GS     & 18 & 1525 & 25208 & 25,74         & 35,37 & 8,27  & 6,23 \\[3ex]
 KER  & Scopus & 7  & 265  & 1773  & 5,43          & 0,83  & 5,53  & 0,67 \\
      & WoS    & 7  & 275  & 1700  & 5,41          & 0,92  & 5,43  & 0,79 \\
      & GS     & 7  & 543  & 2694  & 4,68          & 0,76  & 4,67  & 0,64 \\[3ex]
 KBf  & Scopus & 5  & 211  & 5232  & 11,85         & 10,71 & 13,85 & 8,50 \\
      & WoS    & 5  & 261  & 6011  & 17,78         & 12,74 & 20,85 & 6,81 \\
      & GS     & 5  & 320  & 8510  & 19,59         & 15,42 & 16,51 & 6,48 \\[3ex]
 KAIM & Scopus & 11 & 91   & 615   & 10,60         & 16,71 & 3,92  & 2,14 \\
      & WoS    & 13 & 106  & 1382  & 10,18         & 19,29 & 2,00  & 0,97 \\
      & GS     & 15 & 151  & 616   & 19,01         & 34,50 & 5,95  & 3,03 \\[3ex]
 KOJP & Scopus & 2  & 3    & 0     & 0,58          & 0,35  & 0,58  & 1,50 \\
      & WoS    & 2  & 2    & 0     & 0,33          & 0,00  & 0,33  & 1,00 \\
      & GS     & 2  & 6    & 3     & 2,00          & 1,88  & 2,00  & 1,33 \\[0.5ex]
  \bottomrule 
\end{tabularx}
\end{table}

Nasleduje artimetický priemer a medián počtu dokumentov na autora (graf
porovnania mediánov je na Obrázku \ref{fig:p/a.plot}).  Pre ilustráciu sme
k~hodnotám artimetického priemeru uviedli štandardné odchýlky.  Na mnohých
prípadoch (najmä pre hodnoty indikátorov: dokumenty na autora a citácie na dokument)
je smerodajná odchýlka vyšia než samotný priemer. Čo indikuje nedôverihodnosť
použitia artimetického priemeru na tento účel. Napriek tomu v~literatúre sme sa
stretli s~častejším použitím aritmetického priemeru než mediánu
\citep{Lazaridis2010}.

Podobne ako u~priemerov, k~mediánu sme vypočítali obdobu smerodanej odchylky
tzv.\,MAD\footnote{Absolútna odchýlka mediánu (ang.\,\emph{Median
Absolute Deviation}) je robustná štatistická metóda na zistenie rozptylu dát.
$\mathrm{MAD} = \mathrm{Med}(|X_i - \tilde{X}|)$.  Keďže MAD využíva medián,
je menej náchylný na extrémne vybočujúce hodnoty, a preto sa používa v~
distribúciách, ktoré sa príliš odchyľujú od normálnej
distribúcie.\\\url{http://www.statisticshowto.com/median-absolute-deviation/}}
Rozdiel medzi absolútnou hodnotou mediánu a MAD je nižší
než medzi absolútnou hodnotou priemeru a štandardnej odchýlky znamená, že
medián je lepší ukazovateľom.

\begin{table}
  \centering\small
  \caption[Hodnotenie FPV\,--\,počet citácií na publikáciu a $e$-index]{Scientometrické hodnotenie katedier FPV UCM v~Trnave\,--\,počet citácií na publikáciu a $e$-index.}
\label{tab:2-staff.results}
\begin{tabularx}{\textwidth}{XXcccc@{\hspace{3ex}}cccc}
  \toprule\noalign{\vspace{.3ex}}
           &       & \multicolumn{4}{c}{Počet citácií na publikáciu} &  \multicolumn{4}{c}{$e$-index} \\
 Katedra  & Cit. register & $\bar{x}$      & $\sigma$  & $\tilde{x}$ & MAD  & $\bar{x}$      & $\sigma$  & $\tilde{x}$  & MAD  \\[0.3ex]
  \midrule\noalign{\vspace{.5ex}}
 KB   & Scopus & 7,74        & 7,75  & 5,88  & 3,99  & 9,03    & 9,53  & 7,49  & 4,06  \\
      & WoS    & 7,77        & 7,16  & 6,17  & 3,77  & 9,14    & 9,36  & 7,75  & 5,10  \\
      & GS     & 17,88       & 7,40  & 6,50  & 3,31  & 26,88   & 11,45 & 9,83  & 3,95  \\[3ex]
 KBt  & Scopus & 13,54       & 18,36 & 7,76  & 5,29  & 16,12   & 24,59 & 9,27  & 1,92  \\
      & WoS    & 7,40        & 5,74  & 5,70  & 4,70  & 9,63    & 7,62  & 9,00  & 2,45  \\
      & GS     & 9,93        & 14,14 & 4,27  & 3,33  & 18,96   & 26,42 & 12,45 & 2,20  \\[3ex]
 KCh  & Scopus & 12,45       & 12,03 & 10,97 & 1,92  & 14,70   & 16,23 & 10,82 & 4,42  \\
      & WoS    & 12,62       & 10,91 & 11,81 & 2,98  & 14,94   & 15,74 & 11,86 & 4,52  \\
      & GS     & 11,35       & 10,87 & 10,35 & 3,30  & 17,21   & 18,65 & 13,25 & 6,12  \\[3ex]
 KER  & Scopus & 5,49        & 2,73  & 6,63  & 1,19  & 7,63    & 3,84  & 9,27  & 0,27  \\
      & WoS    & 5,21        & 2,28  & 6,06  & 0,92  & 7,26    & 4,46  & 8,31  & 0,75  \\
      & GS     & 4,53        & 1,80  & 4,93  & 0,82  & 9,60    & 4,64  & 11,96 & 0,53  \\[3ex]
 KBf  & Scopus & 16,58       & 13,35 & 13,34 & 13,34 & 19,52   & 16,86 & 18,65 & 16,45 \\
      & WoS    & 16,44       & 15,49 & 14,68 & 9,73  & 23,99   & 19,66 & 25,16 & 14,00 \\
      & GS     & 18,44       & 15,93 & 15,90 & 14,42 & 25,31   & 23,36 & 21,70 & 21,70 \\[3ex]
 KAIM & Scopus & 3,57        & 5,24  & 0,80  & 0,80  & 5,28    & 7,80  & 1,00  & 1,00  \\
      & WoS    & 3,16        & 5,53  & 0,25  & 0,25  & 4,83    & 9,37  & 0,00  & 0,00  \\
      & GS     & 2,83        & 3,58  & 1,00  & 1,00  & 6,25    & 8,57  & 2,00  & 2,00  \\[3ex]
 KOJP & Scopus & --          & --    & --    & --    & --      & --    & --    & --    \\
      & WoS    & --          & --    & --    & --    & --      & --    & --    & --    \\
      & GS     & 0,50        & 0,00  & 0,25  & 0,25  & 0,50    & 0,71  & 0,50  & 0,50  \\[0.5ex]
  \bottomrule
\end{tabularx}
\end{table}

Tabuľka \ref{tab:2-staff.results} obsahuje počet citácií na dokument (grafické
porovnanie mediánov v~grafe Obrázok \ref{fig:c/p.plot}) a Zhangov $e$-index.
Hodnoty dát Katedry odbornej jazykovej prípravy zo Scopusu a WoS nepočítali,
pretože neboli vôbec citované.  štandardných odchýlok a hodnoty aritmetických
priemerov indikátorov počtu publikácií na autora a počtu citácií na dokument
(Tabuľky \ref{tab:1-staff.results} a \ref{tab:1-staff.results}), spozorujeme, že
v~mnohých prípadoch hodnota štandardnej odchýlky presahuje hodnotu priemeru.  Čo
znamená, že základné indikátory ako počet publikácií na~autora a počet citácií
na dokument nie sú vhodné na scientometrickú analýzu tohoto typu.  Zhangov
$e$-index  je modifikácia $h$-indexu, ktorá je citlivá na veľmi citované články
v~malom súbore publikácií (viď. pokapitola \ref{sec:e-index}).  Na základe
výsledkov Katedry biofyziky (Tabuľka \ref{tab:2-staff.results}) môžeme názorne
vidieť, že mediány $e$-indexu sú rádovo väčšie od ostatných (graficky
znáznornené na Obrázku \ref{fig:e-index.plot}).

Hodnoty citačných indikátorov $h$-index a $g$-index sú uvedené v~tabuľke
\ref{tab:3-staff.results} a mediány sú graficky znázornené na Obrázkoch
\ref{fig:h-index.plot} a \ref{fig:g-index.plot}.  Rozdiel medzi hodnotami
priemeru a mediánu Katedry aplikovanej informatiky a je spôsobený výskytom
niekoľko veľmi citovaných pracovníkov oproti ostatným, ktorí sú minimálne
citovaní.

\citet{Kazakis2015} scientometricky porovnával katedry chemického inžinierstva
troch gréckych univerzít (Atény, Solún a Patra).  V~Tabuľke
\ref{tab:kazakis.results} sme porovnali naše výsledky s~Kazakisovymi. Napriek
tomu, že Gréci čerpali iba z~citačného registru Scopus, my sme porovnali
indikátory zo všetkých použitých citačných registrov. Na prvý pohľad si
všimneme rozdiel v~počte pracovníkov (v~tabuľke označené ako $n$).  Grécke
katedry zahrňujú ďaleko viacej ľudí (katedra v~Aténach dokonca až 72
akademikov) než Katedra chémie s~18 pracovníkmi.  Hodnoty priemerného $h$-index
a $g$-indexu Katedry chémie sú  zhruba o~tretinu nižšie než gréckych katedier.
Veľký rozdiel predstavujú príliš vysoké štandardné odchýlky (ktoré skoro
dosahujú hodnotu priemeru). Pre porovnanie Gréci majú menšie štantardné odchýlky.  Myslíme si, že lepší
indikátorom sú mediány, pretože MAD hodnoty sú nižšie než smerodajné odchýlky.
V~tomto prípade sú mediány $h$-index a $g$-indexu Katedry chémie približne 100\,\%
nižšie než gréckych katedier. , že Katedra chémie má
približne dvakrát menej zamestnancov, tak je výsledok očakávateľný.

\citet{Lazaridis2010} vypočítal priemerný $h$-index gréckych chemických
katedier prestížnych gréckych univerzít (Krétska, Patraska, Solúnska,
Ioanninanska, a Aténska).  Dáta čerpal z~citačného registru WoS, pričom na
výpočet $h$-index používal webové rozhranie WoS, alebo ho počítal ručne. 
V~Tabuľke \ref{tab:foo} sú porovnané priemerné $h$-index katedier
spomenutých gréckych univerzít a naše výsledky pre Katedru chémie. Síce
priemerný $h$-index gréckych katedier je porovnateľný s~KCh (s~výnimkou
Krétskej univerzity), ale dátový súbor je výrazne menší. Zdá sa nám
nepravdepodobné aby taká inštitúcia mala také male množstvo publikácií.  Autor
o~získavaní dát píše iba o~problémoch s~gréckymi menami v~anglickej databáze
WoS. V~neposlednom rade nesmieme zabúdať, že autor neuvádza smerodajnú odchýlku
a je nám dobre známe, že distribúcia citačných indikátorov nemusí byť podobná
normálnemu rozdeleniu. V~prípade distribúcie, ktorá sa príliš líši od normálnej 
distribúcie, sa použitie aritmetického priemeru stáva nevhodným.

Na grafickom porovaní mediánov $h$-indexu (viď podkapitola \ref{sec:h-index}) a
$g$-indexu (viď podkapitola \ref{sec:g-index} katedier FPV môžeme vidieť vyrazný
rozdiel medzi Katedrou biofyziky a ostatnými katedrami, najmä Katedry chémie,
ktorá výrazne zaostáva za KBf napriek najvyšiemu počtu publikácií a citácií
(viď Obrázky \ref{fig:publications.plot} a \ref{fig:citations.plot}).  Tento
trend je možné pozorovať na každom hodnotení FPV pomocou citačných indikátorov
(Obrázky \ref{fig:p/a.plot}, \ref{fig:c/p.plot}, \ref{fig:h-index.plot},
\ref{fig:g-index.plot}, \ref{fig:e-index.plot}, \ref{fig:hc-index.plot}, \ref{fig:hi-index.plot},
\ref{fig:hinorm.plot}, \ref{fig:hm-index.plot}, \ref{fig:awcr.plot}, \ref{fig:aw-index.plot}).

Tabuľka \ref{tab:4-staff.results} ukazuje výsledné hodnoty individuálneho
$h$-indexu ($h_{\mathrm{I}}$-index, viď podkapitola \ref{sec:hi-index}) a jeho
variante pre program Publish or Perish (viď podkapitola \ref{sec:pop})
$h_{\mathrm{I,norm}}$ (viď podkapitola \ref{sec:hinorm}). Tieto indikátory
normujú $h$-index na počet spoluautorov. Rádovo vyšie hodnoty aritmetického
priemeru oboch indikátorov GS je pravdepodobne spôsobený vlastnosťou Google Scholaru:
automaticky zoznam autorov zužuje na 3-5 položiek (viď podkapitola \ref{sec:gs}),
čím umelo zvyšuje hodnotu $h_{\mathrm{I}}$-index a $h_{\mathrm{I,norm}}$.
Rozdiel mediánov jednotlivých katedier je graficky zobrazený na Obrázkoch \ref{fig:hi-index.plot} a
\ref{fig:hinorm.plot}.

V~Tabuľke \ref{tab:5-staff.results} uvádzame prehľad výsledkov citačnej frekvencie váhovanej
podľa veku (AWCR) a $AW$-index (viď podkapitola \ref{sec:aw-index}). Ako názov napovedá
tieto indikátory znižujú váhu starších publikácií. Využíva sa hlavne na porovnávanie
vedcov s~rôznymi akademickým vekom, pretože starší profesor s~bohatou akademickou kariérov, ale
na dôchodku sa javý produktívnejší v~očiach $h$-indexu ako mladý vedec na začiatku kariéry
v~plnej sile. Priemerné hodoty AWCR podliehajú veľkej chybe, preto sme na graf použili
mediány (Obrázky \ref{fig:awcr.plot} a \ref{fig:aw-index.plot}).

Na tabuľke \ref{tab:6-staff.results} sú zobrazené výsledné hodnoty súčasného $h$-indexu
($h^{\mathrm{c}}$-index) a multiautorského $h$-indexu ($h_{\mathrm{m}}$-index).
Súčasný $h$-index je ďalší indikátor, ktorý zahrnuje starnutie článkov (viď podkapitola \ref{sec:hc-index}) a
multiautorský $h$-index znižuje hodnotu, podľa počtu autorov (vid. podkapitola \ref{sec:hm-index}).
Mediány týchto citačných indikátorov pre FPV sú vykreslené na Obrázku \ref{fig:hc-index.plot} a Obrázku \ref{fig:hm-index.plot}.

\begin{SCtable}
  \label{tab:kazakis.results}
\centering\small
  \caption[Porovnanie KCh FPV a kat. chem. inžinierstva  troch gréckych univerzít]
  {Porovnanie citačných indikátorov Katedry chémie FPV a katedier chemického inžinierstva troch gréckych univerzít \citep{Kazakis2015}}
\begin{tabular}{lcccccc}
\toprule\noalign{\vspace{.3ex}}
           & \multicolumn{3}{c}{Katedra chémie FPV} & \multicolumn{3}{c}{\citet{Kazakis2015}}  \\
 Indikátor & Scopus & WoS   & GS    &  Atény     & Solún      & Patra      \\[0.3ex]
\midrule\noalign{\vspace{.5ex}}
 $n$         & 18     & 18    & 18    & 72    & 34    & 30    \\
 $P$         & 1106   & 1120  & 1525  & 4463  & 2253  & 2573  \\
 P/A$^\dagger$         & 15.42  & 16.14 & 25.74 & 62    & 66,3  & 85,8  \\
 $c$         & 19248  & 19223 & 25208 & 74368 & 39695 & 63718 \\
 C/P$^\ddagger$         & 12,45  & 12,62 & 11,35 & 16,7  & 17,6  & 24,8  \\[1ex]
 $\bar{h}$   & 11,78  & 11,94 & 13,61 & 16,3  & 16,8  & 21,3  \\
 $\sigma (h)$ & 11,70  & 11,70 & 12,99 & 8     & 8,5   & 1,5   \\
 $\tilde{h}$ & 7,50   & 7,50  & 9,00  & 15,5  & 17    & 18    \\
 $\bar{g}$   & 19,83  & 20,22 & 23,56 & 26,8  & 28,3  & 35,5  \\
 $\sigma (g)$  & 22,23  & 21,68 & 24,89 & 12,6  & 13,9  & 23    \\
 $\tilde{g}$  & 14,50  & 15,00 & 16,50 & 26,5  & 27    & 30,5  \\[0.5ex]
\bottomrule
  \multicolumn{7}{l}{\footnotesize $^\dagger$ počet autorov na publikáciu; $^\ddagger$ počet citácii na publikáciu} \\
\end{tabular}
\end{SCtable}

\begin{table}
\centering\small
  \caption[Porovnanie KCh FPV a chemických katedier vybraných gréckych univerzít]
  {Porovnanie citačných indikátorov Katedry chémie FPV  a katedier chémie piatich gréckych univerzít \citep{Lazaridis2010}}
\begin{tabularx}{\textwidth}{Xccclccccc}
\toprule\noalign{\vspace{.5ex}}
           & \multicolumn{3}{c}{Katedra chémie FPV}& \phantom{M} & \multicolumn{5}{c}{\citet{Kazakis2015}} \\
Indikátor  & Scopus & WoS   & GS                 &  & Kréta & Patra & Solún & Ioaninna & Atény \\[0.3ex]
\midrule\noalign{\vspace{.5ex}}
 $P$         & 1106   & 1120  & 1525               & & 56    & 61    & 41    & 48       & 219   \\
 $\bar{h}$  & 11,78  & 11,94 & 13,61              & & 16,6  & 12,6  & 10,4  & 10,3     & 9,0   \\[0.5ex]
  \bottomrule
\end{tabularx}
  \label{tab:foo}
\end{table}

\begin{table}
\centering\small
  \caption[Porovnanie KEB, KBt a vybranej skupiny enviromentalistov]
  {Porovnanie citačných indikátorov Katedry ekochémie a rádiobiológie a vybranej skupiny vedcov z~oblasti enviromentalistiky, ktorí sa zúčastnili
  projektu ACUMEN FP7 \citep{Wildgaard2015}.}
\begin{tabularx}{\textwidth}{Xlcccccccccc}
  \toprule\noalign{\vspace{.3ex}}
 kat. &Cit.\,reg.   & $n$  & $p$    & $c$     & A/P$^\dagger$  & C/P$^\ddagger$ & AWCR & AW  & $h$    & $g$    & $e$    \\[0.3ex]
\midrule\noalign{\vspace{.5ex}}
 ENV  & WoS      & 99 & 3228 & 34851 & 3,1   & 7,8   & 42,4  & 5,4  & 8,5  & 13,1  & 9,1   \\
      & GS       & 99 & 7425 & 62351 & 3,2   & 7,6   & 76,1  & 7,5  & 11,9 & 18,4  & 13,2  \\[2ex]
 KER  & WoS      & 7  & 275  & 1700  & 5,41  & 5,21  & 40,24 & 5,82 & 7,57 & 11    & 7,26  \\
      & GS       & 7  & 543  & 2694  & 4,68  & 4,53  & 74,13 & 8,04 & 9,43 & 14,71 & 9,6   \\[2ex]
 KBt  & WoS      & 1  & 351  & 3707  & 7,83  & 7,4   & 33,76 & 4,71 & 6,55 & 12,09 & 9,63  \\
      & GS       & 1  & 828  & 14901 & 23,15 & 14,14 & 84,97 & 7,48 & 9,73 & 22,55 & 18,96 \\[0.5ex]
\bottomrule
    \multicolumn{12}{l}{\footnotesize $^\dagger$ počet autorov na publikáciu; $^\ddagger$ počet citácii na publikáciu} \\
\end{tabularx}
  \label{tab:foo2}
\end{table}

\citet{Wildgaard2015} porovnával 17 citačných indikátorov na vzorke vedcov
z~rôznych vedných oblastí. Jednou z~nich bola enviromentalistika. Vzorka
obsahovala $n$ akademikov od doktorov po profesorov. Porovnávali sme Katedru
ekochémie a rádiobiológie (KER), Katedru biotechnlógií a už vyšie uvedenú vzorku
vedcov.  Porovnanie hodnôt citačných indikátorov je uvedené v~Tabuľke
\ref{tab:foo2}.  Dáta boli získané z~citačných registrov \emph{Web of
Science} (WoS) a \emph{Google Scholar} (GS).  Napriek veľkému rozdielu v~počte
akademikov ($n$), vo väčšine prípadov katedry FPV vykazujú väčšie kvality než
skupina enviromentalistov. Ale treba brať do úvahy, že skupina nie je viazana
afiliáciou, ale účasťou na projekte ACUMEN
FP7\footnote{\url{http://research-acumen.eu/}}.

Vytvorili sme zoznam časopisov s~najväčším počtom $n$ publikácií (zoradené
zostupne) z~citačných registrov WoS a Scopus (viď Príloha).  Tabuľka zahrňuje
najpoužívanejšie citačné indikátory na hodnotenie časopisov: impakt faktor
(IF); Scimago rang časopisov; SciteCore a SNIP (viď Tabuľku
\ref{tab:indicators.review}).

Zo scientometrického hodnotenia vyplýva, že najproduktívnejšia katedra je
Katedra biofyziky, pretože dosahuje najvyšie hodnoty mediánov citačných
indikátorov napriek výraznému rozdielu v~absolútnych hodnotách počtu publikácií
a citácií oproti Katedre chémie. Síce najhoršie sa umiestnila Katedra odbornej
jazykovej prípravy (KOJP). Ako už názov napovedá táto katedra sa zameriava na
prírodné vedy. Podarilo sa nám vyhľadať do desať publikácií s~minimom citácií.
Preto sme ich dáta nevykresľovali do grafov. 

Ďalšia katedra je katedra aplikovanej informatiky a matematiky (KAIM) má na
internetových stránkach uvedený relatívne rozsiahly zoznam pracovníkov (15),
ale publikácie všetkých uvedených pracovníkov sa nám podarilo vyhľadať iba
z~citačného registru Google Scholar (GS), ktorý prehľadáva všetky voľne prístupné
publikácie. Teda má prístup k~najväčšiemu počtu dokumentov. Extrémné hodnoty
smerodajnej dochýlky (napr. v~Tabuľke \ref{tab:5-staff.results}) sú spôsobené
nevyváženou distribúciou citácií, teda valnej väčšine citácií katedry prislúcha
niekoľkým veľmi citovaných pracovníkom. Ostatní nemajú takmer žiadne citácie.

Katedra biológie, biotechnológií, ekochémie a rádiobiológie a dokonca aj chémie
sú zhruba na rovnakej úrovni. Pri niektorých indikátoroch vyniká jedna, či
druhá, ale s~celkovým hodnotením a započítaním chyby nie je možné ďalej určiť
poradie.

\begin{figure}
  \centering
  \includegraphics[width=\textwidth]{plot-results-data-papers.png}
  \caption{Celkový počet publikácií jednotlivých katedier FPV UCM v~Trnave}
  \label{fig:publications.plot}
\end{figure}

\begin{figure}
  \centering
  \includegraphics[width=\textwidth]{plot-results-data-citations.png}
  \caption{Celkový počet citácií jednotlivých katedier FPV UCM v~Trnave}
  \label{fig:citations.plot}
\end{figure}

\begin{figure}
  \centering
  \includegraphics[width=\textwidth]{plot-results-data-papers_author.png}
  \caption{Medián počtu publikácií na autora pre jednotlivé katedry FPV UCM v~Trnave}
  \label{fig:p/a.plot}
\end{figure}


\begin{figure}
  \centering
  \includegraphics[width=\textwidth]{plot-results-data-cites_paper.png}
  \caption{Medián citácií na publikáciu pre jednotlivé katedry FPV UCM v~Trnave}
  \label{fig:c/p.plot}
\end{figure}

\begin{table}
  \centering\small
  \caption[Hodnotenie FPV\,--\,$h$-index a $g$-index]{Scientometrické hodnotenie katedier FPV UCM v~Trnave\,--\,$h$-index a $g$-index.}
\label{tab:3-staff.results}
\begin{tabularx}{\textwidth}{XXcccc@{\hspace{3ex}}cccc}
  \toprule\noalign{\vspace{.3ex}}
       &           & \multicolumn{4}{c}{$h$-index}     & \multicolumn{4}{c}{$g$-index}    \\
  Katedra & Cit. register & $\bar{x}$      & $\sigma$  & $\tilde{x}$ & MAD  & $\bar{x}$      & $\sigma$  & $\tilde{x}$  & MAD  \\[0.3ex]
  \midrule\noalign{\vspace{.5ex}}
 KB   & Scopus & 7,93    & 7,65  & 6,50  & 4,50  & 12,71   & 13,22 & 11,00 & 7,00  \\
      & WoS    & 8,23    & 7,57  & 7,00  & 5,00  & 12,92   & 13,07 & 10,50 & 7,50  \\
      & GS     & 18,88   & 8,39  & 6,00  & 3,50  & 19,88   & 15,19 & 11,50 & 6,00  \\[3ex]
 KBt  & Scopus & 7,45    & 8,07  & 5,00  & 3,00  & 18,36   & 27,16 & 12,00 & 4,00  \\
      & WoS    & 6,55    & 7,38  & 4,00  & 3,00  & 12,09   & 11,04 & 10,00 & 3,00  \\
      & GS     & 9,73    & 8,83  & 7,00  & 4,00  & 22,55   & 28,96 & 16,00 & 4,00  \\[3ex]
 KCh  & Scopus & 11,78   & 11,70 & 7,50  & 4,50  & 19,83   & 22,23 & 14,50 & 9,50  \\
      & WoS    & 11,94   & 11,70 & 7,50  & 4,50  & 20,22   & 21,68 & 15,00 & 9,50  \\
      & GS     & 13,61   & 12,99 & 9,00  & 5,50  & 23,56   & 24,89 & 16,50 & 9,50  \\[3ex]
 KER  & Scopus & 8,00    & 4,40  & 9,00  & 3,00  & 11,71   & 6,55  & 14,00 & 2,00  \\
      & WoS    & 7,57    & 4,35  & 8,00  & 4,00  & 11,00   & 6,88  & 12,00 & 3,00  \\
      & GS     & 9,43    & 4,76  & 11,00 & 2,00  & 14,71   & 7,11  & 18,00 & 2,00  \\[3ex]
 KBf  & Scopus & 12,80   & 10,57 & 15,00 & 10,00 & 24,80   & 21,61 & 26,00 & 20,00 \\
      & WoS    & 15,50   & 12,37 & 16,00 & 8,00  & 30,75   & 25,00 & 32,00 & 17,50 \\
      & GS     & 16,00   & 13,17 & 17,00 & 13,00 & 31,80   & 28,26 & 30,00 & 23,00 \\[3ex]
 KAIM & Scopus & 3,91    & 5,82  & 1,00  & 1,00  & 6,82    & 10,42 & 1,00  & 1,00  \\
      & WoS    & 3,85    & 7,06  & 1,00  & 1,00  & 6,62    & 12,50 & 1,00  & 1,00  \\
      & GS     & 4,33    & 6,00  & 2,00  & 2,00  & 8,13    & 11,28 & 3,00  & 3,00  \\[3ex]
 KOJP & Scopus & --      & --    & --    & --    & --      & --    & --    & --    \\
      & WoS    & --      & --    & --    & --    & --      & --    & --    & --    \\
      & GS     & 1,00    & 0,00  & 0,50  & 0,50  & 1,00    & 0,00  & 0,50  & 0,50  \\[0.5ex]
  \bottomrule
\end{tabularx}
\end{table}

\begin{figure}
  \centering
  \includegraphics[width=\textwidth]{plot-results-data-h_index.png}
  \caption{Medián $h$-indexu pre jednotlivé katedry FPV UCM v~Trnave}
  \label{fig:h-index.plot}
\end{figure}

\begin{figure}
  \centering
  \includegraphics[width=\textwidth]{plot-results-data-g_index.png}
  \caption{Medián $g$-indexu pre jednotlivé katedry FPV UCM v~Trnave}
  \label{fig:g-index.plot}
\end{figure}


\begin{table}
  \centering\small
  \caption[Hodnotenie FPV\,--\,$h_{\mathrm{I}}$-index a $h_{\mathrm{I,norm}}$]{Scientometrické hodnotenie katedier FPV UCM v~Trnave\,--\,$h_{\mathrm{I}}$-index a $h_{\mathrm{I,norm}}$}
\label{tab:4-staff.results}
\begin{tabularx}{\textwidth}{XXcccc@{\hspace{3ex}}cccc}
  \toprule\noalign{\vspace{.3ex}}
        &             & \multicolumn{4}{c}{$h_{\mathrm{I}}$-index}  & \multicolumn{4}{c}{$h_{\mathrm{I,norm}}$}  \\
Katedra & Cit. register& $\bar{x}$      & $\sigma$  & $\tilde{x}$ & MAD  & $\bar{x}$      & $\sigma$  & $\tilde{x}$  & MAD  \\[0.3ex]
  \midrule\noalign{\vspace{.5ex}}
 KB   & Scopus & 1,94     & 2,82 & 1,10 & 0,85 & 4,00    & 4,96 & 2,50 & 1,50 \\
      & WoS    & 2,08     & 2,88 & 1,46 & 0,89 & 4,23    & 5,40 & 3,00 & 2,00 \\
      & GS     & 21,88    & 3,40 & 1,48 & 0,84 & 22,88   & 5,85 & 3,50 & 2,00 \\[3ex]
 KBt  & Scopus & 1,46     & 1,66 & 0,84 & 0,34 & 3,64    & 2,94 & 3,00 & 0,00 \\
      & WoS    & 1,24     & 1,47 & 0,82 & 0,46 & 3,36    & 2,80 & 3,00 & 1,00 \\
      & GS     & 2,41     & 2,15 & 1,73 & 0,71 & 5,00    & 3,74 & 4,00 & 1,00 \\[3ex]
 KCh  & Scopus & 2,10     & 1,64 & 1,55 & 0,86 & 5,17    & 4,72 & 4,00 & 2,00 \\
      & WoS    & 2,20     & 1,77 & 1,53 & 0,94 & 5,50    & 4,82 & 4,00 & 2,00 \\
      & GS     & 3,02     & 2,87 & 2,08 & 1,23 & 6,61    & 6,55 & 5,00 & 2,50 \\[3ex]
 KER  & Scopus & 1,56     & 0,93 & 1,72 & 0,53 & 3,14    & 1,68 & 3,00 & 2,00 \\
      & WoS    & 1,46     & 0,93 & 1,65 & 0,36 & 3,00    & 2,00 & 3,00 & 1,00 \\
      & GS     & 1,88     & 1,02 & 1,94 & 0,54 & 4,14    & 2,04 & 5,00 & 1,00 \\[3ex]
 KBf  & Scopus & 2,49     & 2,29 & 2,50 & 1,88 & 6,80    & 6,06 & 6,00 & 6,00 \\
      & WoS    & 3,17     & 2,86 & 2,91 & 1,73 & 8,00    & 6,73 & 8,00 & 5,00 \\
      & GS     & 3,44     & 2,90 & 3,04 & 2,04 & 8,60    & 6,84 & 8,00 & 6,00 \\[3ex]
 KAIM & Scopus & 1,74     & 2,52 & 0,50 & 0,50 & 2,73    & 3,69 & 1,00 & 1,00 \\
      & WoS    & 1,60     & 2,75 & 0,50 & 0,50 & 2,46    & 4,35 & 1,00 & 1,00 \\
      & GS     & 1,66     & 2,28 & 1,00 & 0,67 & 2,93    & 4,17 & 1,00 & 1,00 \\[3ex]
 KOJP & Scopus & --       & --   & --   & --   & --      & --   & --   & --   \\
      & WoS    & --       & --   & --   & --   & --      & --   & --   & --   \\
      & GS     & 0,67     & 0,47 & 0,50 & 0,34 & 0,50    & 0,71 & 0,50 & 0,50 \\[0.5ex]
  \bottomrule
\end{tabularx}
\end{table}

\begin{figure}
  \centering
  \includegraphics[width=\textwidth]{plot-results-data-hi_index.png}
  \caption{Medián $h_{\mathrm{I}}$-indexu pre jednotlivé katedry FPV UCM v~Trnave}
  \label{fig:hi-index.plot}
\end{figure}

\begin{figure}
  \centering
  \includegraphics[width=\textwidth]{plot-results-data-hi_norm.png}
  \caption{Medián $h_{\mathrm{I,norm}}$ pre jednotlivé katedry FPV UCM v~Trnave}
  \label{fig:hinorm.plot}
\end{figure}


\begin{table}
  \centering\small
  \caption[Hodnotenie FPV\,--\,AWCR a $AW$-index]{Scientometrické hodnotenie katedier FPV UCM v~Trnave\,--\,AWCR a $AW$-index.}
\label{tab:5-staff.results}
\begin{tabularx}{\textwidth}{Xlcccc@{\hspace{3ex}}cccc}
  \toprule\noalign{\vspace{.3ex}}
          &       & \multicolumn{4}{c}{AWCR}         & \multicolumn{4}{c}{$AW$-index}  \\
  Katedra &  Cit. register& $\bar{x}$      & $\sigma$  & $\tilde{x}$ & MAD  & $\bar{x}$      & $\sigma$  & $\tilde{x}$  & MAD  \\[0.3ex]
  \midrule\noalign{\vspace{.5ex}}
 KB   & Scopus & 37,50  & 66,36  & 20,21  & 15,31  & 4,64     & 4,15 & 4,46  & 2,25 \\
      & WoS    & 38,77  & 66,64  & 21,65  & 16,82  & 4,83     & 4,09 & 4,65  & 2,45 \\
      & GS     & 23,88  & 91,79  & 22,06  & 15,15  & 24,88    & 4,70 & 4,69  & 1,69 \\[3ex]
 KBt  & Scopus & 59,31  & 121,61 & 24,57  & 9,57   & 5,93     & 5,15 & 4,96  & 1,09 \\
      & WoS    & 33,76  & 54,44  & 19,41  & 9,83   & 4,71     & 3,57 & 4,41  & 1,32 \\
      & GS     & 84,97  & 153,60 & 37,05  & 17,60  & 7,48     & 5,65 & 6,09  & 1,30 \\[3ex]
 KCh  & Scopus & 122,65 & 217,78 & 40,94  & 38,87  & 8,34     & 7,49 & 6,37  & 2,78 \\
      & WoS    & 120,48 & 207,33 & 44,28  & 39,76  & 8,33     & 7,36 & 6,61  & 3,36 \\
      & GS     & 157,82 & 272,44 & 53,79  & 47,38  & 9,56     & 8,39 & 7,29  & 4,40 \\[3ex]
 KER  & Scopus & 45,53  & 28,61  & 58,01  & 12,18  & 6,24     & 2,78 & 7,62  & 0,76 \\
      & WoS    & 40,24  & 26,21  & 46,23  & 15,52  & 5,82     & 2,73 & 6,80  & 1,06 \\
      & GS     & 74,13  & 45,04  & 86,72  & 19,96  & 8,04     & 3,33 & 9,31  & 1,02 \\[3ex]
 KBf  & Scopus & 85,87  & 79,65  & 88,52  & 76,25  & 7,71     & 5,76 & 9,41  & 3,86 \\
      & WoS    & 104,22 & 92,18  & 101,11 & 67,94  & 8,61     & 6,33 & 9,90  & 3,26 \\
      & GS     & 129,18 & 125,84 & 116,20 & 115,44 & 9,50     & 6,98 & 10,78 & 5,81 \\[3ex]
 KAIM & Scopus & 9,17   & 15,03  & 1,50   & 1,50   & 2,10     & 2,28 & 1,22  & 0,94 \\
      & WoS    & 10,07  & 21,56  & 0,25   & 0,25   & 1,73     & 2,77 & 0,50  & 0,50 \\
      & GS     & 18,00  & 28,49  & 3,25   & 3,25   & 3,09     & 3,01 & 1,80  & 1,80 \\[3ex]
 KOJP & Scopus & --     & --     & --     & --     & --       & --   & --    & --   \\
      & WoS    & --     & --     & --     & --     & --       & --   & --    & --   \\
      & GS     & 0,67   & 0,47   & 0,17   & 0,50   & 0,79     & 0,30 & 0,29  & 0,50 \\[0.5ex]
  \bottomrule
\end{tabularx}
\end{table}

\begin{figure}
  \centering
  \includegraphics[width=\textwidth]{plot-results-data-awcr.png}
  \caption{Medián AWCR pre jednotlivé katedry FPV UCM v~Trnave}
  \label{fig:awcr.plot}
\end{figure}

\begin{figure}
  \centering
  \includegraphics[width=\textwidth]{plot-results-data-aw_index.png}
  \caption{Medián $AW$-indexu pre jednotlivé katedry FPV UCM v~Trnave}
  \label{fig:aw-index.plot}
\end{figure}


\begin{table}
  \centering\small
  \caption[Hodnotenie FPV\,--\,$h^{\mathrm{c}}$-index a $h_{\mathrm{m}}$-index]{Scientometrické hodnotenie katedier FPV UCM v~Trnave\,--\,$h^{\mathrm{c}}$-index a $h_{\mathrm{m}}$-index.}
\label{tab:6-staff.results}
\begin{tabularx}{\textwidth}{XXcccc@{\hspace{3ex}}cccc}
  \toprule\noalign{\vspace{.3ex}}
       &      & \multicolumn{4}{c}{$h^{\mathrm{c}}$-index} & \multicolumn{4}{c}{$h_{\mathrm{m}}$-index} \\
 Katedra      & Cit. register& $\bar{x}$      & $\sigma$  & $\tilde{x}$ & MAD  & $\bar{x}$      & $\sigma$  & $\tilde{x}$  & MAD  \\[0.3ex]
  \midrule\noalign{\vspace{.5ex}}
 KB   & Scopus & 5,14     & 4,42 & 4,50  & 2,00 & 3,81     & 5,07 & 1,75 & 1,75 \\
      & WoS    & 5,46     & 4,56 & 5,00  & 2,00 & 4,04     & 5,23 & 2,88 & 1,68 \\
      & GS     & 20,88    & 5,15 & 6,00  & 2,00 & 27,88    & 5,60 & 2,34 & 1,99 \\[3ex]
 KBt  & Scopus & 5,64     & 3,50 & 5,00  & 1,00 & 3,34     & 3,88 & 1,90 & 0,90 \\
      & WoS    & 4,55     & 3,21 & 4,00  & 1,00 & 2,92     & 3,52 & 1,98 & 0,99 \\
      & GS     & 7,18     & 3,74 & 6,00  & 1,00 & 4,66     & 4,58 & 3,33 & 1,30 \\[3ex]
 KCh  & Scopus & 7,94     & 7,68 & 6,00  & 3,00 & 5,03     & 4,81 & 3,17 & 2,22 \\
      & WoS    & 7,94     & 7,39 & 6,50  & 3,50 & 5,40     & 5,16 & 3,06 & 2,26 \\
      & GS     & 9,44     & 8,49 & 8,00  & 4,00 & 6,26     & 6,22 & 3,59 & 2,45 \\[3ex]
 KER  & Scopus & 7,00     & 2,83 & 8,00  & 1,00 & 3,57     & 2,47 & 3,87 & 1,41 \\
      & WoS    & 6,14     & 2,34 & 7,00  & 1,00 & 3,31     & 2,26 & 3,70 & 0,98 \\
      & GS     & 8,00     & 3,00 & 9,00  & 1,00 & 4,44     & 2,44 & 4,41 & 2,49 \\[3ex]
 KBf  & Scopus & 8,00     & 6,08 & 11,00 & 2,00 & 5,62     & 5,01 & 6,79 & 3,86 \\
      & WoS    & 9,00     & 7,07 & 9,50  & 4,50 & 6,45     & 5,00 & 6,91 & 3,11 \\
      & GS     & 9,80     & 7,60 & 11,00 & 7,00 & 6,49     & 5,32 & 7,09 & 5,01 \\[3ex]
 KAIM & Scopus & 2,09     & 1,97 & 2,00  & 2,00 & 2,56     & 4,01 & 0,83 & 0,67 \\
      & WoS    & 1,46     & 2,15 & 1,00  & 1,00 & 2,45     & 4,44 & 0,50 & 0,50 \\
      & GS     & 2,87     & 2,75 & 2,00  & 2,00 & 2,93     & 4,40 & 1,25 & 1,25 \\[3ex]
 KOJP & Scopus & --       & --   & --    & --   & --       & --   & --   & --   \\
      & WoS    & --       & --   & --    & --   & --       & --   & --   & --   \\
      & GS     & 1,00     & 0,00 & 0,50  & 0,50 & 0,67     & 0,47 & 0,50 & 0,34 \\[0.5ex]
  \bottomrule
\end{tabularx}
\end{table}

\begin{figure}
  \centering
  \includegraphics[width=\textwidth]{plot-results-data-hc_index.png}
  \caption{Medián $h^\mathrm{c}$-indexu pre jednotlivé katedry FPV UCM v~Trnave}
  \label{fig:hc-index.plot}
\end{figure}

\begin{figure}
  \centering
  \includegraphics[width=\textwidth]{plot-results-data-hm_index.png}
  \caption{Medián $h_\mathrm{m}$-indexu pre jednotlivé katedry FPV UCM v~Trnave}
  \label{fig:hm-index.plot}
\end{figure}

\begin{figure}
  \centering
  \includegraphics[width=\textwidth]{plot-results-data-e_index.png}
  \caption{Medián $e$-indexu pre jednotlivé katedry FPV UCM v~Trnave}
  \label{fig:e-index.plot}
\end{figure}


%%% Local Variables:
%%% TeX-master: "diplomovka"
%%% End:
