\chapter{Literárny prehľad}
\setcounter{page}{1}
\pagenumbering{arabic}

\section{Definícia pojmov}

\subsection{Bibliometria}

Termín bibliometria je zložený z dvoch gréckych slov: biblion, čo zamená kniha a
métron, meranie.  Takže doslovný preklad by bol meranie kníh, alebo veda
zaoberajúca sa meraním kníh.  Zrozumiteľnejšia je prvá definícia: aplikácia
matematických a štatistikých metód na knihy a iné komunikačné
média. \citet{Pritchard1969}

V súčasnosti sa pod týmto termínom chápe súhrn štatistickým metód na
kvanitatívnu analýzu publikácií v písomnej forme, ako sú knihy, alebo články vo
forme bibliografických záznamov.  Tieto záznamy zahrňujú informácie ako názov
publikácie, jej autorov, rok publikovania, ale aj kľúčové slová, abstrakt, či
referencie na iné publikácie.  na bibliometrických záznamoch môžeme študovať:

\begin{itemize}
\item aspekty tvorby publikácií -- autori, použitá literatúra,
\item aspekty šírenia publikácií -- komunikačné kanály ako názov časopisu,
\item aspekty použitia publikácií -- citačné prepojenia, ale aj štatistika
  požičiavania v knižnici, alebo frekvencia prístupu cez web.  \citep{Ondrisova2011}
\end{itemize}

Najčastejšia bibliometrická metóda je tzv.  citačná analýza, pri ktorej sa
štatisticky spracovávajú citačné prepojenia na iné dokumenty (citácie).  V nej
sa ďalej zahrňujú ostatné informácie bibliografických záznamov, ako počet
autorov (priemerný počet autorov na dokument, priemerný počet citácií na autora
za dokument), počet strán (priemerný počet citácií na stranu dokumentu), počet
publikácií v konkrétnom časopise a zmeny týchto informácií za isté obdobie.  To
znamená, že analýzou dát z bibliometrických záznamov môžeme sledovať vývoj
jednotlivých oblastí, ich vzájomný vplyv a prepojenia.

Na základe týchto empirických dát sa vytvárajú matematicko-štatistické modely,
ktorými sa snažia opísať procesy súvisiace s tvorbou, šírením a použítím
zaznamenaných informácií.

Bibliometria úzko súvisí s ďalšími disciplínami ako scientometria, informetria,
librametria, webometria a cybermetria.  Všetky tieto disciplíny skúmajú
kvantitátívne aspekty informácií a preto je metodika veľmi podobná, líšia sa iba
oblasťou, ktorú skúmajú.


\subsection{Scientometria}

Termín scientometria môžeme rozdeliť na dve slová: latinské {\em scientia}, čo znamená
poznanie a už spomináne grécke {\em métron}, teda meranie.  doslovne \uv{meranie
poznania.} Pojem scientometria Nalimov (1969) definoval ako kvantitatívne
metódy, ktoré sú používané na analýzu vedeckého poznania a výskumu. 

Scientometriu je možné považovať ako aplikáciu bibliometrie na vedecký výskum a
pokrok.  V súčasnosti sa na kvantifikáciu vedeckého pokroku využívajú vedecké
články.  Lenže ich samotný počet nič nehovorí o ich kvalite.  Indikátorom
kvality vedeckých publikácii sú tzv. citácie.  Teda odkazy na pôvodnú
publikáciu, z ktorej čerpajú.  Ich počet je kvantitatívnym znakom kvality
článku.  Pri analýze niekoľkých článkov, napr. vyprodukovaných jedným
pracovníkom je potrebné zahrňovať distribúciu citácii, medzi článkami.  Na to
slúži tzv. citačná analýza.

Okrem vedeckých publikácií scientometria skúma aj ďalšie kvantitatívne aspekty
vedy ako napr. človekoroky, počet rokov praxe vedcov, finančné vstupy apod.
\citet{Bellis2009}


\subsection{Informetria}

Pod termínom informetria sa chápu kvantitatívne aspekty informácií v ľubovoľnej
forme v ľubovoľnej sociálnej skupine.

Termín sa začal používať až koncom 80-tych rokov ako spoločný názov pre
biblometriu a scientometriu, ale stále sa bibliometria, scientometria a
informetria používajú ako synonymá.


\subsection{Webometria}

S rozvojom informačných technológií a hlavne internetu sa presunula pozornosť na
informácie v prostredí internetu.  Webometria skúma kvantitatívne aspekty
konštrukcie a využívania informačných zdrojov, štruktúr a technológií na webe
čerpajúc z bibliometrických a informetrických prístupov.


\subsection{Cybermetria}

Cybermetria sa na rozdiel od webometrie sa zaoberá kvantitatívnymi aspektami
iných internetových služieb ako sú diskusné skupiny, alebo elektronická pošta.


\subsection{Bibliometrické zákony}

Pod termínom bibliometrický zákon (alebo taktiež nazývaný informetrický zákon)
chápeme matematický model, ktorý opisuje empirické závislosti bibliometrických
dát a javy ako distribúciu dokumentov v istom súbore rôznych autorov, alebo
distribúciu citácií v istom súbore dokumentov apod.  Bibliometrické zákony sú
odvodené ako generalizácia istých štatitstických dát.  (Todeschini et al., 2016)

V období medzi rokmi 1920 a 1930 boli publikované tri hlavné bibiometrické
zákony: Lotkov zákon distribúcie vedeckých prác medzi autormi, Bradfordov zákon
rozdelenia publikácií konkrétneho oboru vo vedeckých časopisoch a Zipfov zákon
distribúcie slov v texte (De Bellis, 2009).


\subsection{Lotkov zákon}

Pomenovaný podľa amerického chemika, matematika a štatistika Alfreda J.  Lotku
opisuje frekvenicu publikácie prác v danom obore vzhľadom na autorov.  Lotka
zoradil autorov podľa počtu publikácií a analyzoval koľko prác prislúcha k
prvému autorovi, druhému atď.  Dáta čerpal z indexov {\em Chemical Abstract} a
{\em Geschichtstafeln der Physik} (Lotka, 1926).  Vyšla mu jednoduchá
matematická závislosť.  Počet autorov $f(n)$, ktorí publikovali $n$ článkov v
danom obore ($n = 1, 2, 3, \dots$) sa blíži ku $1/n^2$ násobku počtu autorov,
ktorí publikovali jeden článok.

Lotkov zákon je matematicky definovaný vzťahom (\ref{eq:lotkov_zakon}), v ktorom
$K$ a~$\alpha$ sú kladné konštanty závisace na vedeckej oblasti.  Vo väčšine
prípadov platí, že $\alpha = 2$ a $K = 1$ (Egghe, 2005).
\begin{equation}
\label{eq:lotkov_zakon}
f(n) = \frac{K}{n^\alpha}
\end{equation}

Ak je známy počet autorov s jedným článkom ($a_1$), je možné pomocou vzťahu
(\ref{eq:lotkov_zakon2}) z Lotkovho zákona určiť približný počet autorov s $n$
publikáciami v danom vedeckom obore.
\begin{equation}
\label{eq:lotkov_zakon2}
a_n = \frac{a_1}{n^2}
\end{equation}
Napríklad v súbore 100 autorov by 4 autori mali mať každý 5 publikácii
($100/5^2 = 4$).


\subsection{Bradfordov zákon}

Britský knihovník Samuel Clement Bradford, si všimol istú pravidelnosť v
distribúcii počtu článkov s konkrétnou tématikou vo vedeckých časopisoch.  V
roku 1934 publikoval prácu, v ktorej popísal tento jav.  V danej vedeckej práci
študoval bibliografické záznamy časopisov z oblasti geofyziky.  Články týkajúce
sa istej témy našiel v 326 časopisoch.  Potom zostupne usporiadal časopisy podľa
počtu článkov spadajúcich to danej témy.  Nakoniec ich rozčlenil do troch skupín
tak, aby každá skupina obsahovala zhruba taký istý počet článkov.  Vyšlo mu:

\begin{itemize}
\item prvá skupina obsahovala 9 časopisov s 429 článkami,
\item druhá skupina obsahovala 59 časopisov s 499 článkami,
\item tretia skupina obsahovala 258 časopisov s 404 článkami.
\end{itemize}

Prvú skupinu s najväčším počtom článkov na časopis pomenoval ako jadro, druhú
pomenoval ako prvú zónu a tretiu pomenoval ako druhú zónu.

Počty časopisov v jednotlivých skupinách dal do pomeru:

\begin{equation}
\label{eq:bradfordov_zakon1}
9 : 59 : 258 \\
a_n = \frac{a_1}{n^2}
\end{equation}

ktorý sa blíži ku:

\begin{equation}
\label{eq:bradfordov_zakon2}
9 : (9 \cdot 5) : (9 \cdot 5^2)
\end{equation}

Teda pomer:

\begin{equation}
\label{eq:bradfordov_zakon3}
9 : 5 : 5^2
\end{equation}

Podľa, ktorého definoval všeobecnú matematickú definíciu ako:

\begin{equation}
\label{eq:bradfordov_zakon4}
1 : n : n^2 : \dotso
\end{equation}

pričom $n$ sa nazýva Bradfordov násobok a je závislý od konkrétnych
bibliometrický dát.

Bradfordov zákon je považovaný za najlepší model vedeckého výskumu knižničnej a
informačnej vedy (Nicolaisen, 2007).


\subsection{Zipfov zákon}

Americký jazykovedec George Kingsley Zipf študoval kvantitatívnu analýzu jazyka.
Konkrétne analyzoval text knihy Odyseus od Jamesa Joycesa.  Vybral z textu 29
899 špecifických slov (vylúčil bežné slová ako predložky, spojky apod.)  a
zoradil ich podľa frekvencie výskytu.  Prvé najvfrekventovanejšie slovo dostalo
rang 1, druhé rang 2, atď.  Potom vynásobil frekvenciu výskytu každého slova s
príslušným rangom.  Prekvapujúco mu vyšli veľmi podobné hodnoty.  Toto zistenie
definoval matematicky ako:

\begin{equation}
\label{eq:zipfov_zakon}
c = r \cdot f
\end{equation}

pričom $r$ je rang (poradové číslo) daného slova a $f$ je frekvencia výskytu
slova v texte.  Tým pádom $c$ je konštanta, ktorá reprezentuje daný text
(Powers, 1998).


Paradoxne Zipfov zákon neplatí iba v lingvistike, ale je ho možné aplikovať v
každej oblasti, kde sa skúma frekvencia výskytu konkrétneho javu.  Ako
napr. distribúcia počtu citácií, alebo návštevnosť webových stránok (Li, 2002).

Zipfov zákon je možné aplikovať na počty obyvateľov v mestách.  V najväčšom
meste je dvojnásobok počtu obyvateľov ako v druhom najväčšom meste a trojnásobok
ako v treťom najväčšom meste (Jiang et. al, 2015).


%%% Local Variables:
%%% TeX-master: "diplomovka"
%%% End:
