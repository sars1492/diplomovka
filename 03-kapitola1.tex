\chapter{Literárny prehľad}
\setcounter{page}{1}
\pagenumbering{arabic}

\section{Definícia pojmov}

\subsection{Bibliometria}

Termín bibliometria\index{bibliometria} je zložený z~dvoch gréckych slov:
\textporson{biblion}, čo zamená kniha, a \textporson{métron}, meranie.  Takže
doslovný preklad by bol \uv{meranie kníh}, alebo \uv{veda zaoberajúca sa meraním
  kníh}.  Zrozumiteľnejšia je prvá definícia: aplikácia matematických a
štatistikých metód na knihy a iné komunikačné média.  \citep{Pritchard1969}

V~súčasnosti sa pod týmto termínom chápe súhrn štatistickým metód na
kvanitatívnu analýzu publikácií v~písomnej forme, ako sú knihy, alebo články vo
forme bibliografických záznamov.  Tieto záznamy zahrňujú informácie ako názov
publikácie, jej autorov, rok publikovania, ale aj kľúčové slová, abstrakt, či
referencie na iné publikácie.  Na bibliometrických záznamoch môžeme študovať:

\begin{itemize}
\item aspekty tvorby publikácií -- autori, použitá literatúra,
\item aspekty šírenia publikácií -- komunikačné kanály ako názov časopisu,
\item aspekty použitia publikácií -- citačné prepojenia, ale aj štatistika
  požičiavania v~knižnici, alebo frekvencia prístupu cez web.
  \citep{Ondrisova2011}
\end{itemize}

Najčastejšia bibliometrická metóda je tzv.  citačná analýza, pri ktorej sa
štatisticky spracovávajú citačné prepojenia na iné dokumenty (citácie).  V~nej
sa ďalej zahrňujú ostatné informácie bibliografických záznamov, ako počet
autorov (priemerný počet autorov na dokument, priemerný počet citácií na autora
za dokument), počet strán (priemerný počet citácií na stranu dokumentu), počet
publikácií v~konkrétnom časopise a zmeny týchto informácií za isté obdobie.  To
znamená, že analýzou dát z~bibliometrických záznamov môžeme sledovať vývoj
jednotlivých oblastí, ich vzájomný vplyv a prepojenia.

Na základe týchto empirických dát sa vytvárajú matematicko-štatistické modely,
ktorými sa snažia opísať procesy súvisiace s~tvorbou, šírením a použítím
zaznamenaných informácií.

Bibliometria úzko súvisí s~ďalšími disciplínami ako scientometria, informetria,
librametria, webometria a cybermetria.  Všetky tieto disciplíny skúmajú
kvantitátívne aspekty informácií a preto je metodika veľmi podobná, líšia sa iba
oblasťou, ktorú skúmajú.


\subsection{Scientometria}

Termín scientometria\index{scientometria} môžeme rozdeliť na dve slová: latinské
\emph{scientia}, čo znamená poznanie a už spomináne grécke \textporson{métron},
teda meranie\,--\,doslovne \uv{meranie poznania.} Pojem scientometria Nalimov
(1969) definoval ako kvantitatívne metódy, ktoré sú používané na analýzu
vedeckého poznania a výskumu.

Scientometriu je možné považovať ako aplikáciu bibliometrie na vedecký výskum a
pokrok.  V~súčasnosti sa na kvantifikáciu vedeckého pokroku využívajú vedecké
články.  Lenže ich samotný počet nič nehovorí o~ich kvalite.  Indikátorom
kvality vedeckých publikácii sú tzv.\,citácie.  Teda odkazy na pôvodnú
publikáciu, z~ktorej čerpajú.  Ich počet je kvantitatívnym znakom kvality
článku.  Pri analýze niekoľkých článkov, napr.\,vyprodukovaných jedným
pracovníkom je potrebné zahrňovať distribúciu citácii, medzi článkami.  Na to
slúži tzv.\,citačná analýza.

Okrem vedeckých publikácií scientometria skúma aj ďalšie kvantitatívne aspekty
vedy ako napr.\,človekoroky, počet rokov praxe vedcov, finančné vstupy apod.
\citet{Bellis2009}


\subsection{Informetria}

Pod termínom informetria\index{informetria} sa chápu kvantitatívne aspekty
informácií v~ľubovoľnej forme v~ľubovoľnej sociálnej skupine.

Termín sa začal používať až koncom 80-tych rokov ako spoločný názov pre
biblometriu a scientometriu, ale stále sa bibliometria, scientometria a
informetria používajú ako synonymá.


\subsection{Webometria}

S~rozvojom informačných technológií a hlavne internetu sa presunula pozornosť na
informácie v~prostredí internetu.  Webometria skúma kvantitatívne aspekty
konštrukcie a využívania informačných zdrojov, štruktúr a technológií na webe
čerpajúc z~bibliometrických a informetrických prístupov.


\subsection{Cybermetria}

Cybermetria sa na rozdiel od webometrie sa zaoberá kvantitatívnymi aspektami
iných internetových služieb ako sú diskusné skupiny, alebo elektronická pošta.


\subsection{Bibliometrické zákony}

Pod termínom bibliometrický zákon (alebo taktiež nazývaný informetrický zákon)
chápeme matematický model, ktorý opisuje empirické závislosti bibliometrických
dát a javy ako distribúciu dokumentov v~istom súbore rôznych autorov, alebo
distribúciu citácií v~istom súbore dokumentov apod.  Bibliometrické zákony sú
odvodené ako generalizácia istých štatitstických dát.  (Todeschini et al., 2016)

V~období medzi rokmi 1920 a 1930 boli publikované tri hlavné bibiometrické
zákony: Lotkov zákon distribúcie vedeckých prác medzi autormi, Bradfordov zákon
rozdelenia publikácií konkrétneho oboru vo vedeckých časopisoch a Zipfov zákon
distribúcie slov v~texte (De Bellis, 2009).


\subsection{Lotkov zákon}

Pomenovaný podľa amerického chemika, matematika a štatistika Alfreda J.\,Lotku
opisuje frekvenicu publikácie prác v~danom obore vzhľadom na autorov.  Lotka
zoradil autorov podľa počtu publikácií a analyzoval koľko prác prislúcha
k~prvému autorovi, druhému atď.  Dáta čerpal z~indexov \emph{Chemical Abstract}
a \emph{Geschichtstafeln der Physik} (Lotka, 1926).  Vyšla mu jednoduchá
matematická závislosť.  Počet autorov $f(n)$, ktorí publikovali $n$ článkov
v~danom obore ($n = 1, 2, 3, \dots$) sa blíži ku $1/n^2$ násobku počtu autorov,
ktorí publikovali jeden článok.

Lotkov zákon je matematicky definovaný vzťahom (\ref{eq:lotkov_zakon}), v~ktorom
$K$ a~$\alpha$ sú kladné konštanty závisace na vedeckej oblasti.  Vo väčšine
prípadov platí, že $\alpha = 2$ a $K = 1$ (Egghe, 2005).
\begin{equation}
\label{eq:lotkov_zakon}
f(n) = \frac{K}{n^\alpha}
\end{equation}

Ak je známy počet autorov s~jedným článkom ($a_1$), je možné pomocou vzťahu
(\ref{eq:lotkov_zakon2}) z~Lotkovho zákona určiť približný počet autorov s~$n$
publikáciami v~danom vedeckom obore.
\begin{equation}
\label{eq:lotkov_zakon2}
a_n = \frac{a_1}{n^2}
\end{equation}
Napríklad v~súbore 100 autorov by štyria autori mali mať každý päť publikácii
($100/5^2 = 4$).


\subsection{Bradfordov zákon}

Britský knihovník Samuel Clement Bradford, si všimol istú pravidelnosť
v~distribúcii počtu článkov s~konkrétnou tématikou vo vedeckých časopisoch.
V~roku 1934 publikoval prácu, v~ktorej popísal tento jav.  V~danej vedeckej
práci študoval bibliografické záznamy časopisov z~oblasti geofyziky.  Články
týkajúce sa istej témy našiel v~326 časopisoch.  Potom zostupne usporiadal
časopisy podľa počtu článkov spadajúcich to danej témy.  Nakoniec ich rozčlenil
do troch skupín tak, aby každá skupina obsahovala zhruba taký istý počet
článkov.  Vyšlo mu:

\begin{itemize}
\item prvá skupina obsahovala 9 časopisov s~429 článkami,
\item druhá skupina obsahovala 59 časopisov s~499 článkami,
\item tretia skupina obsahovala 258 časopisov s~404 článkami.
\end{itemize}

Prvú skupinu s~najväčším počtom článkov na časopis pomenoval ako jadro, druhú
pomenoval ako prvú zónu a tretiu pomenoval ako druhú zónu.

Počty časopisov v~jednotlivých skupinách dal do pomeru:

\begin{equation}
\label{eq:bradfordov_zakon1}
9 : 59 : 258
\end{equation}

ktorý sa blíži ku:

\begin{equation}
\label{eq:bradfordov_zakon2}
9 : (9 \cdot 5) : (9 \cdot 5^2)
\end{equation}

Teda pomer:

\begin{equation}
\label{eq:bradfordov_zakon3}
9 : 5 : 5^2
\end{equation}

Podľa, ktorého definoval všeobecnú matematickú definíciu ako:

\begin{equation}
\label{eq:bradfordov_zakon4}
1 : n : n^2 : \dotso
\end{equation}

pričom $n$ sa nazýva Bradfordov násobok a je závislý od konkrétnych
bibliometrický dát.

Bradfordov zákon je považovaný za najlepší model vedeckého výskumu knižničnej a
informačnej vedy (Nicolaisen, 2007).


\subsection{Zipfov zákon}

Americký jazykovedec George Kingsley Zipf študoval kvantitatívnu analýzu jazyka.
Konkrétne analyzoval text knihy Odyseus od Jamesa Joycesa.  Vybral z~textu
29\,899 špecifických slov (vylúčil bežné slová ako predložky, spojky apod.)  a
zoradil ich podľa frekvencie výskytu.  Prvé najvfrekventovanejšie slovo dostalo
rang 1, druhé rang 2, atď.  Potom vynásobil frekvenciu výskytu každého slova
s~príslušným rangom.  Prekvapujúco mu vyšli veľmi podobné hodnoty.  Toto
zistenie definoval matematicky ako:

\begin{equation}
\label{eq:zipfov_zakon}
c = r \cdot f
\end{equation}

pričom $r$ je rang (poradové číslo) daného slova a $f$ je frekvencia výskytu
slova v~texte.  Tým pádom $c$ je konštanta, ktorá reprezentuje daný text
(Powers, 1998).


Paradoxne Zipfov zákon neplatí iba v~lingvistike, ale je ho možné aplikovať
v~každej oblasti, kde sa skúma frekvencia výskytu konkrétneho javu.  Ako
napr.\,distribúcia počtu citácií, alebo návštevnosť webových stránok (Li, 2002).

Zipfov zákon je možné aplikovať na počty obyvateľov v~mestách.  V~najväčšom
meste je dvojnásobok počtu obyvateľov ako v~druhom najväčšom meste a trojnásobok
ako v~treťom najväčšom meste (Jiang et. al, 2015).


\section{Citačné registre}

Citačné registre\index{Citačné registre} (indexy) sú databázy, z~ktorých je
možné dohľadať citačné odkazy na publikované odborné texty.  Ich analýzou je
možné objektívne posúdiť kvalitu citovaných publikácií.  Citačné registre
vznikli preto, aby bolo možné sledovať, aké ohlasy vo vedeckej komunite vzbudila
daná publikácia.

Z~počiatku citačné indexy vychádzali v~tlačenej forme a ukladané boli ako
mikrofilmy.  S~príchodom nových elektronických médií sa k~nim pridali magnetické
pásky a CD-ROM nosiče.  Od rozšírenia internetu sú všetky citačné registre
prístupné on-line.

\subsection{Web of Science}

Web of Science\index{Citačné registre!Web of Science}\index{Web of Science}
(ďalej len WoS) je online platená služba umožňujúca prístup ku citačným
registrom a ich citačnú analýzu.  Poskytuje komplexné vyhľadávanie vo viacerých
citačných a abstraktových databázach, ktoré umožňuje dôkladne scientometricky
študovať medziodborové oblasti výskumu.  V~minulosti mala názov \emph{Web of
  Knowledge} a bola spravovaná Inštitútom pre vedecké informácie \emph{Institute
  for Scientific Information} (ISI).  V~súčasnosti je vo vlastníctve mediálneho
gigantu \emph{Thomson Reuters} so sídlom New Yorku, USA (Drake, 2005).

WoS pozostáva z~citačných registrov:

\begin{itemize}
\item \textbf{Science Citation Index Expanded\R\ (SCI-E):}\\
  Zahrňuje publikácie z~viac než 8\,500 hlavných časopisov, ktoré pokrývajú 150
  vedeckých disciplín od roku 1900 do súčasnosti.
\item \textbf{Social Sciences Citation Index\R\ (SSCI):}\\
  Články z~viac než 3\,000 časopisov 55 oborov sociálnych vied a vybrané
  publikácie z~3\,500 vo svete najdôležitejších vedeckých a technických
  časopisov od roku 1900 po súčasnosť.
\item \textbf{Arts \& Humanities Citation Index\R\ (A\&HCI):}\\
  Indexuje viac než 1\,700 časopisov z~oblasti umenia a humanitných vied a
  vybrané články z~viac než 250 časopisov z~oblasti sociálnych vied od roku 1975
  do súčasnosti.
\item \textbf{Index Chemicus\R\ (IC):}\\
  Obsahuje viac než 2,6 milióna záznamov zlúčenín od roku 1993.
\item \textbf{Current Chemical Reactions\R\ (CCR):}\\
  Zahrňuje viac než milión chemických reakcií od roku 1986, plus záznamy
  z~francúzkeho Inštitútu duševného vlastníctva (INPI) v~časovom rozmedzí od
  roku 1840 do 1985.
\item \textbf{Book Citation Index\R\ -- Science (BKCI-S) a Book Citation
    Index\R\ -- Social Sciences \& Humanities (BKCI-SSH):}\\
  Poskytuje viac než 50\,000 vybraných kníh.  10\,000 nových kníh pridávaných
  každý rok od 2005 po súčasnosť.
\item \textbf{Conference Proceedings Citation Index\R\ -- Science (CPCI-S) a
    Conference Proceedings Citation Index\R\ -- Social Sciences \& Humanities
    (CPCI-SSH):}\\
  Zahrňuje príspevky z~viac než 160\,000 konferencií z~256 rôznych oblastí
  \uv{vedy a techniky (CPCI-S)} a \uv{sociálnych a humanitných vied (CPCI-SSH)}
  od roku 1990.  Každým rokom do nej pribúda takmer 400\,000 konferenčných
  príspevkov z~cca.\,12\,000
  konferencií.\footnote{\url{http://wokinfo.com/products_tools/multidisciplinary/webofscience/}}
\end{itemize}

Zakladateľom prvého moderného citačného registru sa stal Eugen Garfield (1955).
V~roku 1960 založil inštitúciu \emph{Institute for Scientific Information}
(ISI), ktorá od nasledujúceho roku začala vydávať prvý multidisciplinárny
\uv{Citačný index pre prírodné vedy} \emph{Science Citation Index} (SCI).  Od
roku 1972 sa k~nemu pridal \uv{Citačný index pre sociálne vedy} \emph{Social
  Science Citation Index} (SSCI) a od roku 1978 \uv{Citačný index pre umenie a
  humanitné vedy} \emph{Arts \& Humanities Citation Index} (AHCI), (Smith,
2012).

WoS je časť multidisciplinárnej citačnej databázy \emph{ISI Web of Knowledge},
ktorá je vo vlastníctve spoločnosti \emph{Thomson Reuters}.  Mimo už spomínaného
citačného registru WoS obsahuje:

\begin{itemize}
\item \textbf{Current Contents Connect} -- obsah a bibliografické informácie
z~viac ako 8\,000 vedeckých časopisov,
\item \textbf{Journal Citation Reports} -- ročné bibliometrické hodnotenie a
  porovnávanie vedeckých časopisov,
\item \textbf{Essential Scientific Indicator} -- hodnotenie a porovnávanie
  inštitúcií, krajín a vedných oblastí,
\item \textbf{InCites} -- bibliometrické analýzy a hodnotenia inštitúcií,
  krajín, vedných oblastí,
\item \textbf{Converis} -- komplexný informačný systém výskumnej činnosti
  univerzít,
\item \textbf{ScholarOne} -- manažérsky systém na peer review časopisov,
  konferencií a kníh,
\item \textbf{EndNote} -- komplexný nástroj na bibliometrické
  analýzy.\footnote{\url{http://ipscience.thomsonreuters.com}}
\end{itemize}

V~rozšírenom vyhľadávaní je používateľovi umožnené detailnejšie formulovať
vyhľadávaciu požiadavku pomocou logických operátorov.  Výsledný zoznam
publikácií môže byť usporiadaný podľa roku vydania, relevancie k~danej téme,
počtu citácií, názvu časopisu alebo konferencie.  Výsledky vyhľadávania je možné
dodatočne zúžiť pomocou selekcie konkrétnych predmetových oblastí, typu
dokumentu, autora, roku publikovania a pod.  Ku každému záznamu je zobrazený aj
počet citácií, ktoré publikácia získala, zoznam dokumentov, ktoré ju citovali
alebo zoznam podobných dokumentov na základe bibliografického združovania..  Ak
má používateľ zaplatený prístup, tak je dostupný aj plný text publikácie.
Vyhľadané záznamy je možné exportovať v~rôznych formátoch.  Po exporte sa dajú
ďalej analyzovať pomocou bibliometrického
softvéru.\footnote{\url{http://images.webofknowledge.com/WOKRS57B4/help/WOS/hp_advanced_search.html}}

Dostupné analytické nástroje umožňujú štatisticky vyhodnocovať vyhľadané
záznamy.  Výsledky sú prezentované graficky.

Funkcia \textbf{Analyze Results} ponúka štatistické vyhodnotenie
resp.\,publikačnú analýzu podľa autorov, roku vydania, predmetovej oblasti,
krajiny, jazyka apod.

Funkcia \textbf{Create Citation Report} analyzuje publikácie z~hľadiska počtu
získaných citácií.  Zobrazuje počty publikácií v~jednotlivých rokoch a rovnako
počty citácií, ktoré vybraná množina záznamov získala.  Ak by sme hľadali
publikácie konkrétneho autora, tak tieto grafy zobrazujú jeho publikačnú
produktivitu a úspešnosť v~podobe získaných citácií.  Ku každej množine záznamov
sa zobrazí aj
$h$-index.\footnote{\url{https://images.webofknowledge.com/WOKRS57B4/help/WOS/hp_citation_report.html}}


\subsection{Scopus}

Scopus je citačný register európskeho vydavateľstva \emph{Elsevier} so sídlom
v~Amsterdame, Holansko.  Jedná sa o~platenú službu, rovnako ako WoS.  Bol
spustený v~novembri 2004, ale retrospektívne obsahuje záznamy od roku 1996.
Scopus obsahuje viac záznamov hlavne z~oblasti Európy.  Okrem časopisov obsahuje
aj zborníky, patenty a webové sídla.  Aktualizuje sa denne.  Podľa údajov
z~januára 2016 indexuje viac ako 21\,500 titulov, ktoré zahrňujú:

\begin{itemize}
\item viac než 21\,500 recenzovaných časopisov (z~toho 4\,200 prístupných
  zdarma),
\item viac než 360 obchodných časopisov,
\item viac než 530 knižných edícií,
\item viac než 7,2 milióna konferenčných príspevkov z~83\,000 konferencií
\item viac než 116\,000 knižných titulov 
\item viac ako 27 miliónov patentových záznamov z~piatich patentových úradov
\item články v~tlači (Articles-in-Press) z~viac ako 5\,000 časopisov
\end{itemize}

Databáza k~januáru 2016 obsahuje viac než 60 miliónov záznamov v~jadre, z~toho:

\begin{itemize}
\item viac než 38 miliónov záznamov od roku 1996 (84\,\% všetkých citácií),
\item viac než 22 miliónov záznamov z~obdobia 1823--1995 (staršie záznamy
  obsahujú len abstrakty bez citácií),
\item okolo 3 milióna záznamov pribúda každým rokom (5\,500 za deň).
\end{itemize}

V~decembri 2015 bolo pridaných viac než 93 milióna citácií na viac než päť
miliónov článkov starších z~pred roku
1996.\footnote{\url{https://www.elsevier.com/__data/assets/pdf_file/0007/69451/scopus_content_coverage_guide.pdf}}

Citačný register Scopus má veľmi dobre vyriešenú otázku identifikácie autorov a
inštitúcií.  Každý autor a inštitúcia má priradené všetky formáty mien
resp.\,názvov, zároveň sú k~dispozícii všetky dostupné štatistiky na základe
indexovaných záznamov.

Napriek tomu, že pri možnosti spresnenia výsledkov sú pri jednotlivých
možnostiach zobrazené aj počty záznamov (napríklad počty publikácií
v~jednotlivých rokoch), nie je možné tieto štatistiky prezentovať tak ako
v~databáze WoS.  Funkcia \textbf{View citation overview} umožňuje analyzovať
citácie označených záznamov.  V~prehľadnej tabuľke sa zobrazia publikácie a
citácie v~jednotlivých rokoch.  \textbf{Author Evaluator} je nástroj zobrazujúci
štatistiky a hodnoty indikátorov autorov na základe ich publikácií a získaných
citácií.  Koláčovým grafom sú prezentované časopisy, v~ktorých autor publikoval,
typy dokumentov a predmetové oblasti.  Okrem toho je možné zobraziť počty
publikácií a citácií v~jednotlivých rokoch, všetkých spoluautorov a počty
spoločných publikácií.  Graficky sa zobrazí aj autorov $h$-index.

Funkcia \textbf{Affiliation details} zobrazí detailné informácie týkajúce sa
konkrétnej inštitúcie.  Ide o~počet dokumentov, autorov, názvy časopis,
v~ktorých títo autori publikujú.  Okrem toho sú k~dispozícii údaje o~spolupráci
s~inými inštitúciami a koláčový graf zobrazujúci štruktúru predmetových oblastí
na základe indexovaných publikácií.

\textbf{Journal Analyzer} je nástroj na hodnotenie a porovnávanie vedeckých
časopisov.  Podobnú funkciu má \emph{Journal Citation Reports} v~rámci Web of
Knowledge.  Na hodnotenie sú však zvolené iné indikátory.  Ako alternatíva
k~tzv.\,impakt-faktoru sú uvedené SJR, SNIP, počty citácií v~jednotlivých
rokoch, percento necitovaných článkov a percento prehľadových článkov (reviews).
Hodnoty sú prezentované graficky alebo v~tabuľke.  V~grafe môžu byť na
porovnanie zobrazené údaje o~viacerých
časopisoch.\footnote{\url{https://www.elsevier.com/__data/assets/pdf_file/0005/79196/scopus-quick-reference-guide.pdf}}

Databáza Scopus je silnou konkurenciou pre WoS, resp.\,Web of Knowledge.  Scopus
má širšie obsahové a teritoriálne zameranie, WoS zase dlhšiu tradíciu.  Obidvom
databázam nemožno uprieť snahu o~rozšírenie svojho obsahu a funkcionality,
z~čoho má úžitok hlavne používateľ.


\section{Citačné indikátory}
\label{sec:citation.indicators}

Citačný indikátor\index{Citačné indikátory}38 je druh scientometrickej metódy na
stanovenie \uv{kvality} vedeckých publikácii, vedeckých pracovníkov a vedeckých
inštitúcii.  Všetky indexy vychádzajú zo základných scienometrických parametrov:
počet publikácii a množstvo ich citácii.  Pri výpočte niektorých indexov
zohľadňujú aj iné parametre (napr.\,vek pracovníkov).

Základným indikátorom je citačná frekvencia, t.j.\,priemer počtu citácií istej
skupiny publikácií v~danom obore za určitý
rok.\footnote{\url{http://ipscience-help.thomsonreuters.com/incitesLiveESI/ESIGroup/fieldBaselines/
    citationRatesBaselines.html}}


\subsection{Journal Impact Factor (IF)}

Prvý citačný indikátor navrhol zakladateľ citačných registrov Eugen Garflield
(1955) ako presnejší spôsob evaluácie autorov vedeckých článkov než v~tej dobe
používané počty publikácií a počty citácií.

V~súčastnosti IF používa \emph{Institute of Scientific Information} na
každoročné hodnotenie vedeckých časopisov v~rámci \emph{Journal citation
  reports} (JCR).  Impact Factor je priemerný počet citácií na články
publikované danom časopise za posledné 2 roky.

Impact Factor pre rok 2016 možno matematicky vyjadriť vzťahom (\ref{eq:if}), kde
$a$ je celkový počet článkov, ktoré boli v~danom časopise publikované v~rokoch
2014--2015 a $c$ je počet článkov publikovaných v~danom časopise v~rokoch
2014--2015, ktoré boli citované v~publikáciach indexovaných v~roku
2016.\footnote{Výsledný Impact Factor z~roku 2016 môže byť publikovaný až v~roku
  2017, pretože ho nie je možné vypočítať skôr, než rok 2016 skončí.}

\begin{equation}
\label{eq:if}
\mathit{IF}_{2016} = \frac{c}{a}
\end{equation}

JCR poskytuje IF za 5 ročné obdobie.\footnote{\url{http://admin-apps.webofknowledge.com/JCR/help/h_impfact.htm}}


\subsection{Hirshov index ($h$-index)}

Tento populárny citačný indikátor bol definovaný Jorgem E.\,Hirshom v~roku 2005
ako číslo $h$, ktoré zodpovedá počtu najcitovanejších článkov daného autora,
ktorých každá publikácia má aspoň $h$ citácii.  (Hirsh, 2005)

Pre lepšie pochopenie je vhodné uviesť príklad: Vedec A~má 10 publikácii.  Ak
ich zoradíme podľa počtu citácii, potom prvá má 10 citácii, druhá má 8, tretia
5, štvrtá 4, piata 2 a ostatné nemajú žiadne citácie.  Potom tento vedec má
$h$-index 4, pretože štyri najcitovanejšie články (s~počtami citácii 10, 8, 5 a
4) majú aspoň po 4 citácie.  Tieto najcitovanejšie články sa označujú ako
tzv.\,h-core.

Hlavným problémom $h$-indexu je necitlivosť na malý počet veľmi citovaných článkov
(Napríklad ak porovnáme publikačnú činnosť vedca A~s~predchádzajúceho príkladu
s~vedcom B, ktorý má iba 5 publikácií so 108, 45, 12, 5 a 2 citáciami, jeho
Hirshov index je rovnaký ako vedca A.

Sám Hirsh uviedol, že $h$-index nemožno použiť na porovnávanie autorov rôznych
vedných disciplín.


\subsection{Eggheov index ($g$-index)}

Leo Egghe v~roku 2006 publikoval indikátor $g$-index, ktorý má vyriešiť niektoré
problémy $h$-indexu, najmä jeho necitlivosť k~autorom, ktorí majú málo extrémne
citovaných publikácii (Egghe 2006).

Eggheov $g$-index je definovaný ako číslo $g$, ktoré predstavuje počet
najcitovanejších článkov konkrétneho autora, zostupne zoradený podľa počtu
citácií, ktorého druhá mocnina je menšia alebo rovná súčtu všetkých citácií
daných článkov.

Napríklad ak m8 vedec A~desať publikácií, ktoré majú 6, 6, 5, 4, 2, 0, 0, 0, 0 a
0 citácií.  Jeho $h$-index je 4 a $g$-index je 4.
($6+6+5+4 = 21 \geq 4\cdot4=16$) Vedec B má šesť publikácií s~15, 10, 5, 4, 3 a
2 citáciami.  Aj jeho $h$-index je tiež 4, ale $g$-index je 6.
($15+10+5+4+3+2 = 39 \geq 6\cdot6 = 36$)

Ako je z~príkladu zrejmé, $g$-index $\geq$ $h$-index.  Keďže $g$-index berie do
úvahy viacej citácií, ale stále je necitlivý ku autorom malého počtu extrémne
citovaných článkov.  Pokiaľ vedec má napr.\,10 publikácií, ktoré sú spolu 300
citované, tak jeho maximálny $g$-index je 10 a zvyšných 200 citácií je
ignorovaných.  Z~toho dôvodu sám Egghe navrhol umelo zvýšiť počet článkov na
číslo $T$, ktorého druhá mocnina sa blíži ku celkovému počtu citácií (Egghe,
2008).  Samozrejme manipulovanie s~dátami nie je dobrá metóda a preto je
potrebné použiť iný indikátor, ktorý je schopný zachytiť podobné prípady.


\subsection{Zhangov $e$-index}

Ako reakciu na malú citlivosť $h$-index a $g$-indexu pre autorov s~malým
množstvom veľmi citovaných prác Chun-Ting Zhang navrhol nový indikátor
$e$-index.  Zhang ho definoval ako číslo $e$, ktoré je druhou odmocninou
rozdielu všetkých citácií h-core článkov a maximálnym počtom citácií, ktoré sú
zahrnuté do $h$-indexu $h^2$ (Zhang, 2009).

Napríklad, ak $h$-index akademického pracovníka je 10 a jeho publikácie v~h-core
majú spolu 200 citácií, tak jeho $e$-index bude 10, pretože ak odčítame
teoretické miminum na dosiahnutie $h$-indexu 10, t.j.\,100 citácii od skutočného
počtu citácii h-core článkov 200, výsledok bude 100, z~čoho druhá odmocnina je
10.

To namená, že $e$-index možno použiť na odlíšenie dvoch vedcov s~rovnakým
$h$-indexom, ale rozdielnou citačnou frekvenciou.


\subsection{Súčasný $h$-index (Contemporary $h$-index)}

Autori citačných indikátorov si uvedomujú že vedecká literatúra starne.
Všeobecne vedecká práca z~pred 10 rokov má menší impakt, ako rok stará
publikácia s~rovnakým množstvom citácií.  Sám Hirsh (2005) navrhol
tzv.\,m-kvocient, čo nie je nič iné ako h-index podelený počtom rokov od vydania
prvej práce daného vedca.  Teda výrazne znevýhodňuje starších akademikov, bez
ohľadu na to, či sú stále aktívni, a citovanosť ich najnovších publikácií.

Z~toho dôvodu Sidiropoulos a kol. (2007) navrhli indikátor, ktorý zahrňuje vek
jednotlivých článkov.  Pomenovali ho Súčasný (contemporary) $h$-index
$h^{\mathrm{c}}$, ktorý definovali:

\uv{Vedec má súčasný $h$-index $h^{\mathrm{c}}$ ak každý jeho článok z~množiny
  $\mathrm{N_p}$ dosiahne skóre $S^{\mathrm{c}}(j) \geq h^{\mathrm{c}}$ a ostané
  články $(\mathrm{N_p} - h^{\mathrm{c}})$ dosiahli skóre
  $S^{\mathrm{c}}(j) \geq h^{\mathrm{c}}$.}

Skóre $S^{\mathrm{c}}(j)$ je definované vzťahom (\ref{eq:sc}), pričom $Y(j)$
predstavuje rok, kedy bol článok $j$ publikovaný a $\mathit{cit}_j$ znamená jeho
maximálny počet citácií.

\begin{equation}
\label{eq:sc}
S^{\mathrm{c}}(j) = \gamma\cdot (Y(\mathrm{teraz}) - Y(j) + 1)^{-\delta}\cdot \mathit{cit}_j
\end{equation}

Pri nastavení $\delta = -1$ sa dosiahne, že počet citácií daného článku je
podelený jeho vekom v~rokoch.  Lenže, podľa autorov, podelením počtu citácií
daného článku jeho vekom sa získajú príliš malé hodnoty skóre
$S^{\mathrm{c}}(j)$ na dosiahnutie reprezentatívneho $h$-indexu.  Preto autori
zaviedli koeficient $\gamma$, ktorý podľa empirickej štúdie autorov je
najvýchodnejšie nastaviť na $\gamma = 4$.

\subsection{Citačná frekvencia váhovaná podľa veku (Age-weighted citation rate
  -- AWCR) a $\mathit{AW}$-index}

Vek vedeckej publikácie (rozdiel medzi dnešným rokom a rokom vydania daného
článku) sa považuje za jeden z~faktorov, ktorý definuje impakt článku.  Na jeho
kvantifikáciu je nutné započítať tento parameter do výpočtu.  Jednoduchým
delením počtu citácií danej publikácie jej vekom váhujeme citačnú frekvenciu
podľa veku (publikácie).

Jin a kol (2007) vytvorili indikátor, ktorý váhuje citačnú frekvenciu h-core
článkov podľa veku.  Nazvali ho $\mathit{AR}$-index.

Jeho matematická definícia je vyjadrená vzťahom (\ref{eq:ar}), pričom $h$ je
Hirshov index autora, $\mathit{cit}_j$ je množstvo citácií j-teho
najcitovanejšieho článku a $a_j$ je počet rokov od publikácie j-teho článku.

\begin{equation}
\label{eq:ar}
\mathit{AR} = \sqrt{\sum_{j=1}^h{\frac{\mathit{cit}_j}{a_j}}}
\end{equation}

Autori programu \emph{Publish or Perish} (ďalej len PoP) vytvorili
$\mathit{AW}$-index\,--\,modifikáciu $\mathit{AR}$-indexu.  Narozdiel od
$\mathit{AR}$-indexu, $\mathit{AW}$-index berie do úvahy všetky publikácie, nie
len tie, ktoré sú začlenené v~h-core.


\subsection{Individuálny (Individual) $h$-index}

Batista a kol. (2006) navrhli nový index -- $h_{\mathrm{I}}$, ktorý by bol
multidisciplinárny na rozdiel od h-indexu.  Kďeže jeden z~hlavných rozdielov
medzi vedeckými disciplínami je množstvo vedcov, ktorí pracujú v~danej
disciplíne, Batista a kol definovali $h_{\mathrm{I}}$-index ako podiel
$h$-indexu s~priemerným počtom autorov h-core článkov.

Matematicky ho definovali vzťahom (\ref{eq:hi}), kde $h$ je Hirshov index a
$\langle N_a \rangle = N_a^{(T)} / h$, pričom $N_a^{(T)}$ je celkový počet
autorov (vrátane opakovaní) h-core článkov.

\begin{equation}
\label{eq:hi}
h_{\mathrm{I}} = \frac{h}{\langle N_a \rangle} = \frac{h^2}{N_a^{(T)}}
\end{equation}


\subsection{Individuálny (Individual) $h$-index $h_{\mathrm{I, norm}}$}

Autori programu \emph{Publish or Perish} spravili modifikáciu individuálneho
$h$-indexu $h_{\mathrm{I}}$.  Tento index pomenovali $h_{\mathrm{I, norm}}$.  Na
rozdiel od $h_{\mathrm{I}}$, jednoducho $h$-index delili počtom všetkých
spoluautorov h-core článkov.  Pri výpočte $h_{\mathrm{I, norm}}$ sa počet
citácií jednotlivých článkov podelí počtom autorov daného článku.  A~potom sa
vypočíta Hirshov index z~už takto znormalizovaných publikácií.


\subsection{Multi-autorský $h$-index}

Michael Schreiber (2008) popísal nový indikátor, ktorý zahrňuje spoluautorstvo
-- $h_{\mathrm{m}}$-index.  Schreiber ho odvodil od individuálneho $h$-indexu
$h_{\mathrm{I}}$ s~tým rozdielom, že počtom autorov je delený rang dokumentu,
nie počet citácií, ako v~$h_{\mathrm{I}}$ a z~toho sa vypočíta $h$-index.

Matematicky je definovaný vzťahom (\ref{eq:hm}), pričom $r$ je tzv.\,\emph{rang
  publikácie} v~zostupnom zoradení podľa počtu citácií, $c(r)$ je počet citácií
článku $r$ a $r_{\mathrm{eff}}(r)$ je efektívny rang článku $r$.
\begin{equation}
\label{eq:hm}
h_{\mathrm{m}} = \max_r{(r_{\mathrm{eff}}(r) \leq c(r))}
\end{equation}
Efektívny rang článku $r$ je definovaný vzťahom (\ref{eq:reff}), kde $a(r')$ je
počet autorov publikácie $r'$.
\begin{equation}
\label{eq:reff}
r_{\mathrm{eff}}(r) = \sum_{r'=1}^r{\frac{1}{a(r')}}
\end{equation}


%%% Local Variables:
%%% TeX-master: "diplomovka"
%%% End:
